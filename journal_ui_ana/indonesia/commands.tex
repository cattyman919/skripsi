% journal_ui_ana/indonesia/commands.tex
% Diadaptasi dari settings.tex dan journal_ieee/commands.tex

% --- Informasi Dokumen dari settings.tex (Skripsi) ---
\newcommand{\myJudulIndonesia}{Implementasi dan Analisis Efektivitas Vxlang \textit{Code Virtualization} dalam mempersulit \textit{Reverse Engineering}}
\newcommand{\myPenulis}{Seno Pamungkas Rahman}
\newcommand{\myNPM}{2106731586} % Mungkin tidak diperlukan untuk jurnal, tapi disimpan
\newcommand{\myPembimbing}{Dr. Ruki Harwahyu, S.T., M.T., M.Sc.} % Sesuai contoh PDF Anda (dengan gelar)
\newcommand{\myDepartment}{Departemen Teknik Elektro}
\newcommand{\myFaculty}{Fakultas Teknik}
\newcommand{\myUniversity}{Universitas Indonesia}
\newcommand{\myAddress}{Kampus UI Depok 16424, Jawa Barat, Indonesia}
\newcommand{\myEmail}{seno.pamungkas@ui.ac.id}

\newcommand{\myTitleEnglish}{Implementation and Analysis of the Effectiveness of Vxlang Code Virtualization in Complicating Reverse Engineering} % Dari settings.tex \judulInggris, sedikit disesuaikan kapitalisasinya agar umum untuk judul

\newcommand*{\Assets}{../../assets/pics}%
