%
% Halaman Abstrak
%
% @author  Andreas Febrian
% @version 1.00
%

\chapter*{Abstrak}

\vspace*{0.2cm}
{
	\setlength{\parindent}{0pt}

	\begin{tabular}{@{}l l p{10cm}}
		Nama          & : & \penulis    \\
		Program Studi & : & \program    \\
		Judul         & : & \judul      \\
		Pembimbing    & : & \pembimbing \\
	\end{tabular}

	\bigskip
	\bigskip

Rekayasa balik merupakan ancaman serius terhadap keamanan perangkat lunak, memungkinkan penyerang untuk menganalisis, memahami, dan memodifikasi kode program tanpa izin. Teknik obfuscation, terutama virtualisasi kode, menjadi solusi yang menjanjikan untuk melindungi perangkat lunak dari ancaman ini. Penelitian ini bertujuan untuk mengimplementasikan dan menganalisis efektivitas virtualisasi kode dalam meningkatkan keamanan perangkat lunak dengan mempersulit rekayasa balik. Penelitian ini menggunakan VxLang sebagai platform virtualisasi kode. Metode penelitian yang digunakan meliputi implementasi virtualisasi kode pada sebuah aplikasi studi kasus, kemudian dilakukan analisis statis dan dinamis terhadap aplikasi sebelum dan sesudah di-obfuscate. Analisis statis dilakukan dengan membandingkan tingkat kesulitan dalam memahami kode assembly yang dihasilkan. Analisis dinamis dilakukan dengan mengukur waktu eksekusi dan sumber daya yang digunakan oleh aplikasi. Hasil penelitian menunjukkan bahwa virtualisasi kode dengan VxLang efektif dalam meningkatkan keamanan perangkat lunak. Kode yang telah di-obfuscate menjadi lebih sulit dipahami dan dianalisis, terlihat dari meningkatnya kompleksitas kode assembly. Penelitian ini diharapkan dapat membuktikan bahwa virtualisasi kode dengan VxLang merupakan teknik yang efektif untuk melindungi perangkat lunak dari rekayasa balik dan dapat dipertimbangkan sebagai solusi untuk meningkatkan keamanan aplikasi.

	\bigskip

	Kata kunci:\\
	Pengaburan Kode, Virtualisasi Kode, Perlindungan Perangkat Lunak, Rekayasa Balik
}

\newpage
