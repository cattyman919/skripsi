%
% Halaman Abstract
%
% @author  Andreas Febrian
% @version 1.00
%

\chapter*{Abstract}

\vspace*{0.2cm}
{
	\setlength{\parindent}{0pt}

	\begin{tabular}{@{}l l p{10cm}}
		Name          & : & \penulis      \\
		Study Program & : & \program      \\
		Title         & : & \judulInggris \\
		Counsellor    & : & \pembimbing   \\
	\end{tabular}

	\bigskip
	\bigskip

Reverse engineering is a serious threat to software security, allowing attackers to analyze, understand, and modify program code without permission. Obfuscation techniques, especially code virtualization, are promising solutions to protect software from this threat. This study aims to implement and analyze the effectiveness of code virtualization in improving software security by complicating reverse engineering. This study uses VxLang as a code virtualization platform. The research methods used include implementing code virtualization on a case study application, then conducting static and dynamic analysis of the application before and after obfuscation. Static analysis is done by comparing the level of difficulty in understanding the resulting assembly code. Dynamic analysis is done by measuring the execution time and resources used by the application. The results of the study show that code virtualization with VxLang is effective in improving software security. Obfuscated code becomes more difficult to understand and analyze, as seen from the increasing complexity of the assembly code. This study is expected to prove that code virtualization with VxLang is an effective technique to protect software from reverse engineering and can be considered as a solution to improve application security.

	\bigskip

	Key words:\\
	Code Obfuscation, Code Virtualization, Software Protection, Reverse Engineering
}

\newpage
