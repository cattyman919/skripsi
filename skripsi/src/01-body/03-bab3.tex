%-----------------------------------------------------------------------------%
\chapter{\babTiga}
%-----------------------------------------------------------------------------%
Bab ini menyajikan kerangka kerja metodologis yang digunakan untuk mencapai tujuan penelitian, yaitu mengimplementasikan dan menganalisis efektivitas \f{code virtualization} menggunakan VxLang dalam meningkatkan keamanan perangkat lunak terhadap \f{reverse engineering}. Pembahasan mencakup pendekatan penelitian, desain eksperimen, objek studi, instrumen dan bahan penelitian, prosedur pengumpulan data yang direncanakan termasuk ilustrasi alur kerja, serta teknik analisis data.

%-----------------------------------------------------------------------------%
\section{Pendekatan Penelitian}
%-----------------------------------------------------------------------------%
Penelitian ini menggunakan pendekatan \bo{eksperimental} dengan kombinasi metode \bo{kuantitatif} dan \bo{kualitatif}. Pendekatan eksperimental dipilih karena penelitian ini bertujuan untuk mengukur dan membandingkan efek dari suatu perlakuan (intervensi), yaitu penerapan \f{code virtualization} menggunakan VxLang, terhadap variabel dependen (tingkat kesulitan \f{reverse engineering} dan performa perangkat lunak).
\begin{itemize}
    \item Metode \bo{kuantitatif} akan digunakan untuk mengukur dampak virtualisasi pada performa perangkat lunak (waktu eksekusi dan ukuran file) serta mencatat keberhasilan/kegagalan upaya \textit{bypass} autentikasi sebagai indikator.
    \item Metode \bo{kualitatif} akan digunakan untuk menganalisis dan mendeskripsikan tingkat kesulitan dalam melakukan \f{reverse engineering} (analisis statis dan dinamis) pada kode sebelum dan sesudah virtualisasi, berdasarkan observasi dan interpretasi peneliti.
\end{itemize}

%-----------------------------------------------------------------------------%
\section{Desain Eksperimen}
%-----------------------------------------------------------------------------%
Desain eksperimen yang digunakan adalah \bo{perbandingan antara kelompok kontrol dan kelompok eksperimen} pada serangkaian objek studi.
\begin{itemize}
    \item \bo{Kelompok Kontrol:} Aplikasi studi kasus dan \f{benchmark} yang dikompilasi secara normal tanpa penerapan \f{code virtualization} VxLang.
    \item \bo{Kelompok Eksperimen:} Aplikasi studi kasus dan \f{benchmark} yang sama, namun bagian kode kritikalnya telah diproses menggunakan \f{code virtualization} VxLang.
\end{itemize}
Variabel dalam penelitian ini adalah:
\begin{itemize}
    \item \bo{Variabel Independen:} Penerapan \f{code virtualization} VxLang (diterapkan vs tidak diterapkan).
    \item \bo{Variabel Dependen:}
        \begin{itemize}
            \item Tingkat kesulitan \f{reverse engineering} logika autentikasi.
            \item Performa perangkat lunak (waktu eksekusi dan ukuran file).
        \end{itemize}
\end{itemize}
Eksperimen akan difokuskan pada dua aspek utama: analisis keamanan pada fungsi autentikasi dan analisis \f{overhead} performa pada tugas komputasi spesifik.

%-----------------------------------------------------------------------------%
\section{Objek Studi}
%-----------------------------------------------------------------------------%
Objek studi yang akan dikembangkan dan dianalisis dalam penelitian ini meliputi:
\begin{enumerate}
    \item \bo{Aplikasi Studi Kasus Autentikasi:} Aplikasi simulasi login dalam tiga varian antarmuka (Konsol, Qt, Dear ImGui) dan dua mekanisme autentikasi (Kredensial \textit{hardcoded}, Kredensial via \textit{backend cloud}).
    \item \bo{Aplikasi Benchmark Performa:} Aplikasi untuk mengukur \textit{overhead} pada tugas spesifik (Algoritma QuickSort, Enkripsi AES-CBC-256, Ukuran File Dasar).
    \item \bo{Studi Kasus Aplikasi Kompleks (Lilith RAT):} Sebuah \textit{Remote Administration Tool} (RAT) \textit{open-source} bernama Lilith \cite{LilithRAT} digunakan sebagai studi kasus tambahan. Tujuannya adalah untuk mengevaluasi efektivitas VxLang pada aplikasi yang lebih kompleks dengan beragam fungsionalitas, menguji integritas fungsional setelah virtualisasi, menganalisis kesulitan \textit{reverse engineering}-nya, serta mengamati dampaknya terhadap deteksi oleh layanan pemindaian \textit{malware}.
\end{enumerate}
Setiap objek studi (termasuk klien Lilith RAT) akan dibuat dalam versi asli (non-\textit{virtualized}) dan versi \textit{virtualized} oleh VxLang untuk perbandingan. Server Lilith RAT akan digunakan dalam versi aslinya untuk pengujian fungsionalitas klien.

%-----------------------------------------------------------------------------%
\section{Instrumen dan Bahan Penelitian}
%-----------------------------------------------------------------------------%
Penelitian ini direncanakan menggunakan instrumen dan bahan berikut:
\begin{itemize}
    \item \bo{Perangkat Keras:} Komputer berbasis Windows.
    \item \bo{Perangkat Lunak Pengembangan:} Compiler C++ (Clang/clang-cl), Build System (CMake, Ninja), Library/Framework (Qt, Dear ImGui, OpenSSL, libcurl, dll.), VxLang SDK, Teknologi Backend (Go, PostgreSQL, Docker).
    \item \bo{Instrumen Analisis Keamanan:} Alat Analisis Statis (Ghidra), Alat Analisis Dinamis (x64dbg).
    \item \bo{Instrumen Pengukuran Performa:} Library C++ (\code{std::chrono}, \code{std::filesystem}).
    \item \bo{Lembar Observasi:} Untuk pencatatan kualitatif.
\end{itemize}

%-----------------------------------------------------------------------------%
\section{Prosedur Pengumpulan Data}
%-----------------------------------------------------------------------------%
Pengumpulan data akan mengikuti prosedur terstruktur yang mencakup studi literatur, persiapan artefak, pelaksanaan pengujian, dan pencatatan hasil. Alur kerja umum penelitian ini diilustrasikan pada Gambar \ref{fig:flow_research_steps}.

% --- TikZ Diagram: Alur Penelitian Umum ---
\begin{figure}[H] % Menggunakan [H] dari paket float (via uithesis.sty)
    \centering
    \begin{tikzpicture}[
        scale=0.85, transform shape,
        node distance=3cm and 1cm,
        >=latex,
        block/.style={rectangle, draw, thick, fill=blue!10, text width=9em, text centered, rounded corners, minimum height=3em},
        proc/.style={rectangle, draw, thick, fill=purple!10, text width=10em, text centered, rounded corners, minimum height=3em},
        io/.style={trapezium, trapezium left angle=70, trapezium right angle=110, draw, thick, fill=orange!10, text centered, minimum height=2.5em, text width=10em},
        term/.style={ellipse, draw, thick, fill=gray!20, text centered, minimum height=3em},
        line/.style={draw, thick, -latex'}
    ]
        % Nodes
        \node [block] (start) {Mulai: Studi Literatur \& Perumusan Masalah};
        \node [block, below of=start] (design) {Desain Penelitian (Eksperimen, Objek Studi, Instrumen)};
        \node [proc, below of=design] (develop) {Pengembangan Artefak (Aplikasi Asli \& VM)};
        \node [io, below left=0.8cm and 1.5cm of develop] (test_sec) {Pengujian Keamanan (Analisis Statis \& Dinamis, Upaya Bypass)};
        \node [io, below right=0.8cm and 1.5cm of develop] (test_perf) {Pengujian Performa (Waktu Eksekusi \& Ukuran File)};
        \node [proc, below=3cm of develop] (analyze) {Analisis Data (Kualitatif \& Kuantitatif, Perbandingan, Trade-off)};
        \node [term, below of=analyze] (conclude) {Kesimpulan \& Saran};

        % Paths
        \path [line] (start) -- (design);
        \path [line] (design) -- (develop);
        \path [line] (develop) -- (test_sec);
        \path [line] (develop) -- (test_perf);
        \path [line] (test_sec) -- (analyze);
        \path [line] (test_perf) -- (analyze);
        \path [line] (analyze) -- (conclude);

    \end{tikzpicture}
    \caption{Diagram Alur Umum Tahapan Penelitian.}
    \label{fig:flow_research_steps}
\end{figure}
% --- Akhir TikZ Diagram ---

\subsection{Studi Literatur}
Tahap awal meliputi peninjauan literatur terkait \f{reverse engineering}, teknik \f{obfuscation}, \f{code virtualization} (khususnya VxLang), analisis statis/dinamis, dan metodologi pengukuran performa untuk membangun landasan teori dan memahami penelitian terkait.

\subsection{Persiapan Artefak}
Tahap ini berfokus pada implementasi teknis objek studi. Ini mencakup pengembangan kode sumber untuk aplikasi autentikasi dan \textit{benchmark} performa, integrasi VxLang SDK ke dalam kode, konfigurasi sistem \textit{build} (CMake) untuk menghasilkan versi asli dan versi \textit{intermediate} (dengan penanda VxLang), serta proses kompilasi. Versi \textit{intermediate} kemudian diproses langsung menggunakan \textit{tool command-line} \texttt{vxlang.exe}, yang secara otomatis menghasilkan \textit{executable} akhir tervirtualisasi dengan nama \texttt{(nama\_target)\_vxm.exe}. Detail implementasi disajikan pada Bab 4.

\subsection{Pengujian Keamanan Autentikasi}
Pengujian ini bertujuan untuk mengevaluasi efektivitas VxLang dalam mempersulit analisis dan manipulasi logika autentikasi. Prosedur yang akan diikuti untuk setiap aplikasi autentikasi (asli dan \textit{virtualized}) diilustrasikan pada Gambar \ref{fig:flow_auth_test_proc} dan mencakup langkah-langkah berikut:
\begin{itemize}
    \item \bo{Analisis Statis (Ghidra):} Memuat \textit{executable}, mencari \textit{string} relevan, menganalisis \textit{disassembly/decompilation} untuk memahami alur kontrol, mengidentifikasi instruksi kondisional kritis, dan mencoba melakukan \textit{patching} statis untuk \textit{bypass} autentikasi.
    \item \bo{Analisis Dinamis (x64dbg):} Menjalankan \textit{executable} dalam \textit{debugger}, mencari \textit{string/pattern} saat \textit{runtime}, mengatur \textit{breakpoint}, mengamati alur eksekusi dan nilai memori/register, serta mencoba melakukan manipulasi \textit{runtime} (misalnya, \textit{patching on-the-fly} atau mengubah \textit{flags}/nilai) untuk \textit{bypass} autentikasi.
    \item \bo{Pencatatan Observasi:} Mencatat temuan kualitatif mengenai tingkat kesulitan pada setiap langkah analisis (pencarian \textit{string}, pemahaman alur, identifikasi logika) dan mencatat keberhasilan atau kegagalan setiap upaya \textit{bypass} (statis dan dinamis).
\end{itemize}

% --- TikZ Diagram: Alur Pengujian Autentikasi ---
\begin{figure}[H] % Menggunakan [H] dari paket float (via uithesis.sty)
    \centering
    \begin{tikzpicture}[
        scale=0.8, transform shape,
        node distance=2.5cm and 1.2cm,
        >=latex,
        block/.style={rectangle, draw, thick, fill=blue!10, text width=9em, text centered, rounded corners, minimum height=3em},
        proc/.style={rectangle, draw, thick, fill=purple!10, text width=11em, text centered, rounded corners, minimum height=3em},
        term/.style={ellipse, draw, thick, fill=gray!20, text centered, minimum height=2.5em},
        input/.style={rectangle, draw, thick, fill=green!10, text centered, rounded corners=5pt, minimum height=2.5em, text width=8em},
        line/.style={draw, thick, -latex'}
    ]
        % Nodes
        \node [input] (start) {Mulai (Ambil Executable Auth)}; % Diedit
        \node [proc, below of=start] (static) {Analisis Statis (Ghidra: String, Alur Kontrol, Coba Patch)};
        \node [proc, below of=static] (dynamic) {Analisis Dinamis (x64dbg: Run, Breakpoint, Amati, Coba Manipulasi)};
        \node [block, below of=dynamic] (observe) {Catat Observasi (Kesulitan, Keberhasilan Bypass)};
        \node [term, below of=observe] (end) {Selesai};

        % Paths
        \path [line] (start) -- (static);
        \path [line] (static) -- (dynamic);
        \path [line] (dynamic) -- (observe);
        \path [line] (observe) -- (end);

    \end{tikzpicture}
    \caption{Diagram Alur Prosedur Pengujian Keamanan Autentikasi.}
    \label{fig:flow_auth_test_proc}
\end{figure}
% --- Akhir TikZ Diagram ---
Prosedur ini akan diulang untuk semua varian aplikasi autentikasi (Konsol, Qt, ImGui dalam versi \textit{hardcoded} dan \textit{cloud}), baik untuk versi asli maupun versi \textit{virtualized}.

\subsection{Pengujian Performa Overhead}
Pengujian ini bertujuan mengukur dampak kuantitatif VxLang pada kinerja eksekusi dan ukuran file. Prosedur yang akan diikuti untuk setiap aplikasi \textit{benchmark} (asli dan \textit{virtualized}) diilustrasikan pada Gambar \ref{fig:flow_perf_test_proc} dan melibatkan:
\begin{itemize}
    \item \bo{Pengukuran Waktu Eksekusi:} Menjalankan \textit{benchmark} (QuickSort dan AES) berulang kali (N=100 untuk QuickSort, pemrosesan \textit{batch} 1GB untuk AES) dan mencatat waktu eksekusi menggunakan \code{std::chrono::high\_resolution\_clock}.
    \item \bo{Pengukuran Ukuran File:} Mengukur ukuran \textit{byte} dari file \textit{executable} akhir menggunakan \code{std::filesystem::file\_size} atau utilitas OS.
    \item \bo{Pencatatan Data:} Mencatat semua data waktu eksekusi (setiap \textit{run} dan total/\textit{average}) dan ukuran file untuk analisis selanjutnya.
\end{itemize}

% --- TikZ Diagram: Alur Pengujian Performa ---
\begin{figure}[H] % Menggunakan [H] dari paket float (via uithesis.sty)
    \centering
    \begin{tikzpicture}[
        scale=0.8, transform shape,
        node distance=2cm and 1.2cm,
        >=latex,
        block/.style={rectangle, draw, thick, fill=blue!10, text width=9em, text centered, rounded corners, minimum height=3em},
        proc/.style={rectangle, draw, thick, fill=purple!10, text width=11em, text centered, rounded corners, minimum height=3em},
        term/.style={ellipse, draw, thick, fill=gray!20, text centered, minimum height=2.5em},
        input/.style={rectangle, draw, thick, fill=green!10, text centered, rounded corners=5pt, minimum height=2.5em, text width=8em},
        line/.style={draw, thick, -latex'}
    ]
        % Nodes
        \node [input] (start) {Mulai (Ambil Executable Benchmark)};
        \node [proc, below of=start] (run_time) {Jalankan Benchmark (N kali / Batch)};
        \node [proc, below of=run_time] (measure_time) {Ukur Waktu Eksekusi (std::chrono)};
        \node [proc, below of=measure_time] (measure_size) {Ukur Ukuran File Executable};
        \node [block, below of=measure_size] (record) {Catat Data Waktu \& Ukuran};
        \node [term, below of=record] (end) {Selesai};

        % Paths
        \path [line] (start) -- (run_time);
        \path [line] (run_time) -- (measure_time);
        \path [line] (measure_time) -- (measure_size);
        \path [line] (measure_size) -- (record);
        \path [line] (record) -- (end);

    \end{tikzpicture}
    \caption{Diagram Alur Prosedur Pengujian Performa.}
    \label{fig:flow_perf_test_proc}
\end{figure}
% --- Akhir TikZ Diagram ---
Prosedur ini akan diulang untuk semua aplikasi \textit{benchmark} (QuickSort, Encryption, Size), baik untuk versi asli maupun versi \textit{virtualized}.

\subsection{Pengujian Studi Kasus Lilith RAT}
\label{subsec:prosedur_lilith_rat}
Pengujian terhadap Lilith RAT bertujuan untuk mengevaluasi dampak virtualisasi VxLang pada aplikasi yang lebih kompleks dari segi fungsionalitas, kesulitan analisis, dan deteksi otomatis. Prosedur pengumpulan data untuk Lilith RAT mencakup:
\begin{enumerate}
    \item \bo{Persiapan Artefak Lilith RAT:} Meliputi kompilasi kode sumber klien Lilith versi asli dan versi \textit{intermediate} yang telah ditandai makro VxLang, diikuti pemrosesan dengan \textit{tool} VxLang untuk menghasilkan \textit{executable} klien tervirtualisasi. Detail implementasi disajikan pada Bab 4 (Sub-bab \ref{sec:implementasi_lilith}).
    \item \bo{Pengujian Fungsionalitas:} Menjalankan klien Lilith tervirtualisasi dan server Lilith asli pada dua mesin terpisah dalam satu jaringan lokal. Skenario pengujian meliputi verifikasi koneksi, kemampuan menerima dan mengeksekusi perintah dari server (misalnya, akses \textit{command prompt} jarak jauh, enumerasi direktori, dan pembacaan isi berkas pada mesin klien). Observasi dicatat untuk memastikan fungsionalitas inti RAT tetap berjalan normal setelah virtualisasi.
    \item \bo{Analisis Keamanan (Statis dan Dinamis):} Sama seperti pada aplikasi autentikasi, klien Lilith RAT versi asli dan tervirtualisasi akan dianalisis menggunakan Ghidra (statis) dan x64dbg (dinamis). Fokusnya adalah pada tingkat kesulitan untuk memahami alur kerja utama, mengidentifikasi fungsi-fungsi kritis (misalnya, komunikasi jaringan, pemrosesan perintah, \textit{keylogging}), dan potensi untuk memodifikasi perilaku. Catatan observasi kualitatif akan dibuat.
    \item \bo{Analisis Deteksi VirusTotal:} Kedua versi \textit{executable} klien Lilith (asli dan tervirtualisasi) akan diunggah ke layanan VirusTotal. Hasil pemindaian dari berbagai \textit{engine} antivirus akan dicatat dan dibandingkan, meliputi jumlah deteksi, label ancaman, dan kategori ancaman yang diberikan.
\end{enumerate}
Data yang dikumpulkan dari pengujian Lilith RAT akan dianalisis secara kualitatif (untuk fungsionalitas dan kesulitan \textit{reverse engineering}) dan kuantitatif (untuk perbandingan hasil VirusTotal).

%-----------------------------------------------------------------------------%
\section{Teknik Analisis Data}
%-----------------------------------------------------------------------------%
Data yang telah dikumpulkan akan dianalisis menggunakan teknik berikut:
\begin{itemize}
    \item \bo{Data Kualitatif (Keamanan):} Analisis deskriptif dan interpretatif berdasarkan catatan observasi untuk membandingkan tingkat kesulitan \f{reverse engineering} antara kelompok kontrol dan eksperimen.
    \item \bo{Data Kuantitatif (Performa):} Perhitungan statistik deskriptif (rata-rata, standar deviasi), perhitungan \textit{overhead} waktu eksekusi (persentase), dan perhitungan peningkatan ukuran file. Data akan disajikan dalam tabel dan grafik perbandingan.
    \item \bo{Analisis Trade-off:} Sintesis hasil analisis keamanan dan performa untuk mengevaluasi keseimbangan antara peningkatan proteksi dan dampak pada kinerja.
\end{itemize}
Hasil analisis ini akan menjadi dasar penarikan kesimpulan.
