%-----------------------------------------------------------------------------%
\chapter{\babSatu}
%-----------------------------------------------------------------------------%
% \todo{tambahkan kata-kata pengantar bab 1 disini}


%-----------------------------------------------------------------------------%
\section{Latar Belakang}
%-----------------------------------------------------------------------------%
Perkembangan pesat teknologi perangkat lunak telah mendorong terciptanya aplikasi yang semakin kompleks dan canggih, menawarkan berbagai inovasi dan manfaat di berbagai sektor kehidupan. Namun, kemajuan ini juga diiringi oleh peningkatan ancaman keamanan yang semakin beragam dan canggih. Salah satu ancaman yang signifikan adalah rekayasa balik (reverse engineering). Reverse engineering adalah proses menganalisis suatu sistem, dalam hal ini perangkat lunak, untuk mengidentifikasi komponen-komponennya, interaksi antar komponen, dan memahami cara kerja sistem tersebut tanpa akses ke dokumentasi asli atau kode sumber. Dalam konteks perangkat lunak, reverse engineering memungkinkan pihak yang tidak berwenang untuk membongkar kode program, memahami algoritma yang digunakan, menemukan kerentanan, mencuri rahasia dagang, melanggar hak cipta, dan bahkan menyisipkan kode berbahaya. \\

Teknik-teknik keamanan konvensional seperti enkripsi data dan proteksi password seringkali tidak cukup ampuh untuk mencegah reverse engineering. Enkripsi hanya melindungi data saat transit atau saat disimpan, tetapi tidak melindungi kode program itu sendiri. Penyerang yang berhasil mendapatkan akses ke program yang berjalan dapat mencoba untuk membongkar dan menganalisis kode meskipun data dienkripsi. Demikian pula, proteksi password hanya membatasi akses awal ke program, tetapi tidak mencegah reverse engineering setelah program dijalankan. Penyerang dapat mencoba untuk melewati mekanisme otentikasi atau membongkar program untuk menemukan password atau kunci enkripsi. \\

Oleh karena itu, dibutuhkan teknik perlindungan yang lebih robust dan proaktif untuk mengamankan perangkat lunak dari reverse engineering. Salah satu pendekatan yang menjanjikan adalah obfuscation. Obfuscation bertujuan untuk mengubah kode program menjadi bentuk yang lebih sulit dipahami oleh manusia, tanpa mengubah fungsionalitasnya. Obfuscation dapat dilakukan pada berbagai tingkatan, mulai dari mengubah nama variabel dan fungsi menjadi nama yang tidak bermakna, hingga mengubah alur kontrol program menjadi lebih kompleks dan sulit dilacak. \\

Di antara berbagai teknik obfuscation, code virtualization dianggap sebagai salah satu yang paling efektif. Code virtualization bekerja dengan menerjemahkan kode mesin asli (native code) menjadi instruksi virtual (bytecode) yang dieksekusi oleh mesin virtual (VM) khusus yang tertanam dalam aplikasi \cite{Ore06}. VM ini memiliki Instruction Set Architecture (ISA) yang unik dan berbeda dari ISA prosesor standar. Dengan demikian, tools reverse engineering konvensional seperti disassembler dan debugger tidak dapat langsung digunakan untuk menganalisis kode yang divirtualisasi. Penyerang harus terlebih dahulu memahami ISA dan implementasi VM untuk dapat menganalisis bytecode, yang secara signifikan meningkatkan kompleksitas dan waktu yang dibutuhkan untuk melakukan reverse engineering. \\

VxLang merupakan salah satu platform code virtualization yang menarik untuk dikaji. VxLang menyediakan framework untuk melakukan code virtualization pada berbagai platform dan arsitektur prosesor \cite{VxLang}. Penelitian ini akan mengkaji implementasi dan efektivitas VxLang dalam melindungi perangkat lunak dari rekayasa balik. Dengan menganalisis tingkat kesulitan reverse engineering pada kode yang dilindungi oleh VxLang, ditinjau dari segi analisis statis dan dinamis, penelitian ini diharapkan dapat memberikan kontribusi dalam pengembangan teknik perlindungan perangkat lunak yang lebih aman, handal, dan efektif dalam menghadapi ancaman reverse engineering. Hasil penelitian ini juga diharapkan dapat memberikan informasi berharga bagi para pengembang perangkat lunak dalam memilih dan mengimplementasikan teknik perlindungan yang tepat untuk aplikasi mereka.
%-----------------------------------------------------------------------------%
\section{Rumusan Masalah}
%-----------------------------------------------------------------------------%
Berdasarkan latar belakang di atas, rumusan masalah dalam penelitian ini dirumuskan sebagai berikut:
\begin{enumerate}
	\item Bagaimana implementasi code virtualization menggunakan VxLang pada perangkat lunak, termasuk tahapan-tahapan yang terlibat dan konfigurasi yang diperlukan?
	\item Seberapa efektifkah code virtualization menggunakan VxLang dalam meningkatkan keamanan perangkat lunak terhadap rekayasa balik, diukur dari segi kompleksitas analisis kode menggunakan teknik analisis statis dan dinamis?

	\item Bagaimana pengaruh code virtualization menggunakan VxLang terhadap performa perangkat lunak, ditinjau dari waktu eksekusi, ukuran file program dan apa trade-off antara keamanan dan performa?

\end{enumerate}

%-----------------------------------------------------------------------------%
\section{Tujuan Penelitian}
%-----------------------------------------------------------------------------%
Tujuan dari penelitian ini adalah:

\begin{enumerate}
	\item Mengimplementasikan code virtualization menggunakan VxLang pada sebuah aplikasi studi kasus, mencakup seluruh tahapan implementasi dan konfigurasi.
	\item Menganalisis efektivitas code virtualization menggunakan VxLang dalam meningkatkan keamanan perangkat lunak terhadap reverse engineering melalui analisis statis dan dinamis, membandingkan tingkat kesulitan analisis kode sebelum dan sesudah di-obfuscate.

	\item Mengevaluasi pengaruh code virtualization menggunakan VxLang terhadap performa perangkat lunak dengan mengukur waktu eksekusi, ukuran file program, dan menganalisis trade-off antara keamanan dan performa.

\end{enumerate}

%-----------------------------------------------------------------------------%
\section{Batasan Masalah}
%-----------------------------------------------------------------------------%
Untuk menjaga fokus dan kedalaman penelitian, batasan masalah dalam penelitian ini adalah:
\begin{enumerate}

	\item Platform code virtualization yang digunakan hanya VxLang

	\item Analisis reverse engineering dibatasi pada analisis statis menggunakan disassembler dan decompiler (Ghidra), serta analisis dinamis menggunakan debugger (x64dbg).

	\item Pengujian performa perangkat lunak dibatasi pada pengukuran waktu eksekusi, dan ukuran file program.

\end{enumerate}
%-----------------------------------------------------------------------------%
\section{Metodologi Penelitian}
%-----------------------------------------------------------------------------%
Metode penelitian yang digunakan dalam penelitian ini adalah metode eksperimental kuantitatif. Langkah-langkah penelitian meliputi:
\begin{enumerate}
	\item Studi Literatur
	      Mencakup peninjauan sumber-sumber seperti jurnal, artikel, buku, dan dokumentasi terkait perangkat lunak, rekayasa balik, obfuscation, code virtualization, dan VxLang. Studi literatur ini bertujuan untuk membangun landasan teori yang kuat dan memahami penelitian terdahulu yang relevan.
	\item Konsultasi
	      Melibatkan diskusi berkala dengan dosen pembimbing untuk mendapatkan bimbingan, arahan, dan masukan terkait perkembangan penelitian.
	\item Pengujian Perangkat Lunak
	      Tahap ini meliputi implementasi code virtualization menggunakan VxLang pada aplikasi serta pengujian perangkat lunak untuk mengevaluasi efektivitas dan dampaknya. Pengujian ini dilakukan dengan nmenguji tingkat kesulitan reverse engineering pada aplikasi sebelum dan sesudah di-obfuscate, baik melalui analisis statis (disassembler, decompiler) maupun analisis dinamis (debugger).
	\item Analisis
	      Menganalisis data yang diperoleh dari tahap pengujian perangkat lunak untuk mengevaluasi efektivitas VxLang dalam mempersulit  rekayasa balik dan mengukur dampaknya terhadap performa aplikasi. Analisis ini meliputi perbandingan hasil pengujian antara aplikasi sebelum dan sesudah di-obfuscate.
	\item Kesimpulan
	      Merumuskan kesimpulan akhir dari penelitian berdasarkan hasil analisis data. Kesimpulan harus menjawab rumusan masalah dan tujuan penelitian.
\end{enumerate}

%-----------------------------------------------------------------------------%
\section{Sistematika Penulisan}
%-----------------------------------------------------------------------------%
Seminar ini disusun dengan sistematika sebagai berikut:

BAB 1 – PENDAHULUAN \\
Bab ini berisi latar belakang, rumusan masalah, tujuan penelitian, batasan masalah, metodologi penelitian, dan sistematika penulisan.

BAB 2 – TINJAUAN PUSTAKA \\
Bab ini membahas teori-teori dasar tentang perangkat lunak, rekayasa balik, obfuscation, code virtualization, dan VxLang.

BAB 3 – METODE PENELITIAN \\
Bab ini menjelaskan langkah-langkah penelitian, desain eksperimen, alat dan bahan, serta teknik analisis data.

BAB 4 – HASIL DAN PEMBAHASAN \\
Bab ini menyajikan hasil pengujian dan analisis, serta pembahasan terkait temuan penelitian.

BAB 5 – KESIMPULAN DAN SARAN \\
Bab ini merumuskan kesimpulan dan saran dari penelitian.
