%-----------------------------------------------------------------------------%
\chapter{\babEnam}
%-----------------------------------------------------------------------------%

\section{Kesimpulan}
%-----------------------------------------------------------------------------%
Berdasarkan implementasi dan analisis yang telah dilakukan terhadap efektivitas \f{code virtualization} menggunakan VxLang untuk mempersulit \f{reverse engineering}, dapat ditarik kesimpulan sebagai berikut:

\begin{enumerate}
    \item \textbf{Implementasi VxLang:} Implementasi \f{code virtualization} menggunakan VxLang melibatkan penandaan bagian kode sumber yang kritikal menggunakan makro SDK (\code{VL\_VIRTUALIZATION\_BEGIN/END}), kompilasi menjadi \textit{executable intermediate} (\texttt{*\_vm.exe}) yang tertaut dengan library VxLang, dan pemrosesan \textit{executable intermediate} tersebut secara langsung menggunakan \textit{tool command-line} \texttt{vxlang.exe} untuk menghasilkan \textit{executable} akhir yang tervirtualisasi (\texttt{*\_vxm.exe}). Proses ini mengubah kode \textit{native} yang ditandai menjadi \f{bytecode} yang dieksekusi oleh \textit{virtual machine} (VM) internal VxLang. \textbf{Penelitian ini juga menemukan bahwa penempatan makro virtualisasi secara strategis dan hati-hati sangat krusial, karena penempatan yang kurang tepat pada struktur kode tertentu, terutama yang melibatkan alur kontrol kompleks atau operasi I/O, dapat merusak fungsionalitas aplikasi, sebagaimana teramati pada studi kasus Lilith RAT.}

    \item \textbf{Efektivitas Keamanan:} \f{Code virtualization} menggunakan VxLang terbukti \textbf{efektif secara signifikan} dalam meningkatkan kesulitan \f{reverse engineering} baik secara statis maupun dinamis.
        \begin{itemize}
            \item Pada \textbf{analisis statis} (menggunakan Ghidra), kode yang divirtualisasi menunjukkan hilangnya \f{string-string} penting, pengurangan drastis data terdefinisi, munculnya banyak operasi \textit{assembly} yang tidak dikenali ('???'), dan abstraksi alur kontrol logika inti (seperti perbandingan dan lompatan kondisional pada fungsi autentikasi), sehingga identifikasi dan \textit{patching} langsung menjadi sangat sulit.
            \item Pada \textbf{analisis dinamis} (menggunakan x64dbg), meskipun instruksi \textit{native} dari \textit{Virtual Machine} (VM) VxLang dapat diobservasi setelah aplikasi dimuat, pelacakan alur logika aplikasi asli dan upaya manipulasi \textit{runtime} untuk \textit{bypass} autentikasi tetap sangat terhambat. \textit{String-string} krusial yang telah divirtualisasi (misalnya, pesan status autentikasi seperti "Authentication Failed") \textbf{tidak dapat ditemukan} melalui metode pencarian standar di dalam \textit{debugger}. Logika keputusan inti aplikasi telah terabstraksi ke dalam mekanisme eksekusi VM, sehingga upaya untuk mengidentifikasi dan memodifikasi instruksi \textit{jump} kondisional secara langsung—yang efektif pada versi non-virtualisasi—menjadi \textbf{tidak berhasil} pada kode yang dilindungi VxLang. Akibatnya, manipulasi \textit{runtime} untuk melewati logika (seperti autentikasi) menjadi jauh lebih kompleks dan memerlukan pemahaman mendalam tentang cara kerja VM VxLang.
        \end{itemize}

    \item \textbf{Dampak Performa dan Ukuran File:} Penerapan VxLang memperkenalkan \textbf{\textit{overhead} performa yang substansial} dan \textbf{peningkatan ukuran file yang signifikan}.
        \begin{itemize}
            \item Waktu eksekusi untuk tugas komputasi intensif seperti algoritma QuickSort dan enkripsi/dekripsi AES-CBC-256 meningkat drastis (ratusan hingga puluhan ribu persen untuk QuickSort, dan beberapa ratus persen untuk AES dengan penurunan \textit{throughput} yang tajam).
            \item Ukuran file \f{executable} meningkat secara signifikan, terutama pada program kecil (bisa lebih dari 10 kali lipat), kemungkinan besar karena penyertaan \textit{interpreter} VM VxLang dan \f{bytecode}.
        \end{itemize}

    \item \textbf{Trade-off Keamanan vs. Performa:} Terdapat \textit{trade-off} yang jelas antara peningkatan keamanan terhadap \f{reverse engineering} yang ditawarkan oleh VxLang dengan penurunan kinerja eksekusi dan penambahan ukuran file. VxLang memberikan lapisan proteksi yang kuat namun dengan biaya performa yang tidak dapat diabaikan.

    \item \textbf{Pengaruh terhadap Deteksi Malware:} Studi kasus menggunakan aplikasi RAT Lilith dan sembilan sampel \textit{malware/PUA} lainnya menunjukkan bahwa virtualisasi kode dengan VxLang memiliki dampak yang \textbf{bervariasi} terhadap tingkat deteksi oleh layanan pemindaian \textit{malware} VirusTotal. Pada sekitar separuh sampel yang diuji, jumlah vendor yang mendeteksi sampel sebagai berbahaya menurun setelah virtualisasi, seringkali dengan perubahan karakteristik deteksi dari spesifik menjadi lebih generik atau berbasis heuristik/AI. Namun, pada separuh sampel lainnya, jumlah deteksi justru \textbf{meningkat} setelah virtualisasi. Hal ini mengindikasikan bahwa meskipun VxLang dapat mengaburkan \textit{signature} statis asli, proses virtualisasi itu sendiri atau artefak yang dihasilkannya dapat dikenali sebagai mencurigakan oleh beberapa \textit{engine} antivirus. Secara keseluruhan, virtualisasi VxLang tidak menjamin penghindaran deteksi total dan dampaknya sangat bergantung pada sampel spesifik serta sensitivitas \textit{engine} pendeteksi.
\end{enumerate}

Secara keseluruhan, penelitian ini menunjukkan bahwa \f{code virtualization} dengan VxLang adalah teknik yang ampuh untuk melindungi perangkat lunak dari analisis dan manipulasi oleh \textit{reverse engineer}, namun pengembang perlu mempertimbangkan dampaknya terhadap performa dan ukuran aplikasi secara cermat.

%-----------------------------------------------------------------------------%
\section{Saran}
%-----------------------------------------------------------------------------%
Berdasarkan kesimpulan dari penelitian ini, berikut adalah beberapa saran yang dapat diberikan:

\begin{enumerate}
    \item \textbf{Bagi Pengembang Perangkat Lunak:}
        \begin{itemize}
            \item \textbf{Virtualisasi Selektif:} Mengingat \textit{overhead} performa yang signifikan, disarankan untuk menerapkan \textit{code virtualization} VxLang secara selektif hanya pada bagian kode yang paling kritikal dan sensitif (misalnya, mekanisme validasi lisensi, algoritma inti yang merupakan kekayaan intelektual, fungsi anti-cheat, atau bagian dari logika autentikasi yang kompleks), bukan pada keseluruhan aplikasi.
            \item \textbf{Evaluasi Performa Mendalam:} Sebelum mengimplementasikan VxLang pada produk komersial, lakukan pengujian performa yang menyeluruh pada kasus penggunaan nyata aplikasi tersebut untuk memastikan dampaknya masih dapat diterima oleh pengguna akhir dan tidak mengganggu fungsionalitas secara signifikan.
            \item \textbf{Investigasi Batasan dan Interaksi Penempatan Makro VxLang:} Melakukan penelitian lebih mendalam mengenai batasan-batasan teknis terkait penempatan makro VxLang. Ini dapat mencakup identifikasi jenis-jenis konstruksi kode, pola alur kontrol, atau interaksi dengan API sistem/jaringan yang rentan menyebabkan masalah fungsionalitas ketika divirtualisasi. Hasilnya dapat berupa panduan atau heuristik yang lebih baik bagi pengembang dalam menerapkan VxLang.
            \item \textbf{Kombinasi Teknik:} Pertimbangkan untuk mengkombinasikan VxLang dengan teknik \textit{obfuscation} atau proteksi lainnya (seperti anti-debugging, enkripsi data, atau \textit{code flattening} pada bagian yang tidak divirtualisasi) untuk menciptakan lapisan keamanan yang lebih mendalam (\textit{defense-in-depth}).
        \end{itemize}

    \item \textbf{Untuk Penelitian Selanjutnya:}
        \begin{itemize}
            \item \textbf{Analisis Keamanan Lebih Lanjut:} Melakukan analisis terhadap ketahanan VxLang menggunakan teknik \textit{reverse engineering} yang lebih canggih atau \textit{tool deobfuscation} otomatis yang mungkin dikembangkan khusus untuk menargetkan proteksi berbasis VM.
            \item \textbf{Optimasi Performa:} Meneliti kemungkinan untuk mengurangi \textit{overhead} performa yang disebabkan oleh VxLang, misalnya dengan menganalisis dan mengoptimalkan \textit{interpreter} VM internalnya, atau mengeksplorasi arsitektur VM yang berbeda.
            \item \textbf{Keamanan Interpreter VM:} Menganalisis potensi kerentanan pada \textit{interpreter} VM VxLang itu sendiri yang mungkin dapat dieksploitasi untuk membongkar atau memanipulasi \f{bytecode}.
            \item \textbf{Studi Komparatif:} Membandingkan efektivitas keamanan dan \textit{overhead} performa VxLang dengan solusi \textit{code virtualization} atau \textit{obfuscation} komersial maupun \textit{open-source} lainnya (misalnya, VMProtect, Themida, LLVM Obfuscator).
            \item \textbf{Dukungan Platform Lain:} Melakukan analisis serupa jika/ketika VxLang telah mendukung platform lain seperti Linux ELF atau arsitektur ARM, untuk melihat apakah efektivitas dan dampaknya konsisten.
            \item \textbf{Pengaruh Konfigurasi:} Menyelidiki lebih lanjut pengaruh berbagai opsi konfigurasi yang ditawarkan oleh VxLang (jika ada, seperti tingkat virtualisasi/obfuscation yang berbeda, fitur anti-tamper) terhadap tingkat kesulitan \textit{reverse engineering} dan \textit{overhead} performa.
            \item \textbf{Analisis Mendalam terhadap Interaksi VxLang dengan Berbagai Jenis Malware/PUA:} Mengingat hasil deteksi VirusTotal yang bervariasi, penelitian lebih lanjut diperlukan untuk memahami faktor-faktor spesifik pada \textit{malware/PUA} (misalnya, bahasa pemrograman, teknik \textit{packing} awal, target API) yang menyebabkan penurunan atau justru peningkatan deteksi setelah virtualisasi VxLang. Ini dapat membantu mengidentifikasi skenario di mana VxLang paling efektif atau paling berisiko terdeteksi.
\item \textbf{Evaluasi Risiko Deteksi sebagai "Packed/Protected Software":} Pengembang yang menggunakan VxLang untuk proteksi perangkat lunak sah perlu menyadari potensi bahwa aplikasi mereka dapat ditandai sebagai mencurigakan atau "\textit{RiskWare}" oleh beberapa \textit{engine} antivirus hanya karena penggunaan lapisan virtualisasi yang agresif, meskipun tidak ada niat jahat.
            \item \textbf{Pengembangan Teknik Deteksi untuk Kode Tervirtualisasi:} Menyelidiki pengembangan teknik atau \textit{tool} baru yang dapat lebih efektif mendeteksi atau bahkan melakukan \textit{deobfuscation} parsial terhadap kode yang telah dilindungi oleh \textit{virtualizer} seperti VxLang, terutama dari perspektif analisis \textit{malware}.
        \end{itemize}
\end{enumerate}
