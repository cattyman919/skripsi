%-----------------------------------------------------------------------------%
% Informasi Mengenai Dokumen
%-----------------------------------------------------------------------------%
% 
% Judul laporan. 
\var{\judul}{Implementasi dan Analisis Efektivitas \f{Code Virtualization} dalam Meningkatkan Keamanan \f{Software} dengan Mempersulit \f{Reverse Engineering} menggunakan VxLang}
% 
% Tulis kembali judul laporan, kali ini akan diubah menjadi huruf kapital
\Var{\Judul}{Implementasi dan Analisis Efektivitas \f{Code Virtualization} dalam Meningkatkan Keamanan \f{Software} dengan Mempersulit \f{Reverse Engineering} menggunakan VxLang}
% 
% Tulis kembali judul laporan namun dengan bahasa Ingris
\var{\judulInggris}{Implementation and Analysis of the Effectiveness of Code Virtualization in Improving Software Security by complicating Reverse Engineering}
% 
% Tipe laporan, dapat berisi Skripsi, Tugas Akhir, Thesis, atau Disertasi
\var{\type}{Skripsi}
% 
% Tulis kembali tipe laporan, kali ini akan diubah menjadi huruf kapital
\Var{\Type}{Skripsi}
% 
% Tulis nama penulis 
\var{\penulis}{Seno Pamungkas Rahman}
% 
% Tulis kembali nama penulis, kali ini akan diubah menjadi huruf kapital
\Var{\Penulis}{Seno Pamungkass Rahman}
% 
% Tulis NPM penulis
\var{\npm}{2106731586}
% 
% Tuliskan Fakultas dimana penulis berada
\Var{\Fakultas}{Teknik}
\var{\fakultas}{Teknik}
% 
% Tuliskan Program Studi yang diambil penulis
\Var{\Program}{Teknik Komputer}
\var{\program}{Teknik Komputer}
% 
% Tuliskan tahun publikasi laporan
\Var{\bulan}{Januari}
\Var{\tahun}{2025}
% 
% Tuliskan gelar yang akan diperoleh dengan menyerahkan laporan ini
\var{\gelar}{Sarjana Teknik}
% 
% Tuliskan tanggal pengesahan laporan, waktu dimana laporan diserahkan ke 
% penguji/sekretariat
\var{\tanggalPengesahan}{Januari 2025}
% 
% Tuliskan tanggal keputusan sidang dikeluarkan dan penulis dinyatakan 
% lulus/tidak lulus
\var{\tanggalLulus}{Januari 2025}
% 
% Tuliskan pembimbing 
\var{\pembimbing}{Dr. Ruki Harwahyu, S.T., M.T., M.Sc.}
% 
% Tuliskan penguji
\var{\pengujisatu}{Dr. Ruki Harwahyu, ST. MT. MSc.}
\var{\pengujidua}{I Gde Dharma Nugraha, S.T., M.T., Ph.D}
% 
% Alias untuk memudahkan alur penulisan paa saat menulis laporan
\var{\saya}{Penulis}

%-----------------------------------------------------------------------------%
% Judul Setiap Bab
%-----------------------------------------------------------------------------%
% 
% Berikut ada judul-judul setiap bab. 
% Silahkan diubah sesuai dengan kebutuhan. 
% 
\Var{\kataPengantar}{Kata Pengantar}
\Var{\babSatu}{Pendahuluan}
\Var{\babDua}{Tinjauan Pustaka}
\Var{\babTiga}{Metode Penelitian}
\Var{\babEmpat}{Implementasi dan Hasil Penelitian}
\Var{\babLima}{Kesimpulan}
% \Var{\babEnam}{Bab Enam}
% \Var{\kesimpulan}{Kesimpulan dan Saran}

