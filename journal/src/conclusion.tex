\section{Conclusion} \label{sec:conclusion}
This paper investigated the effectiveness of code virtualization using the VxLang framework as a technique to mitigate software reverse engineering. Through experimental analysis involving static (Ghidra) and dynamic (x64dbg) examination of authentication applications, and performance benchmarking (QuickSort, AES), we draw the following conclusions:

VxLang's code virtualization significantly enhances software security by substantially increasing the difficulty of reverse engineering. The transformation into custom bytecode rendered standard static analysis tools ineffective at interpreting program logic and control flow within virtualized sections. Dynamic analysis was similarly obstructed by the VM execution model, making runtime tracing and manipulation arduous. Attempts to bypass authentication logic, which were trivial in non-virtualized versions, were successfully thwarted in the virtualized binaries using the employed techniques.

However, this robust security comes with significant drawbacks. We observed substantial performance overhead, with execution times for computational tasks increasing dramatically (by factors ranging from hundreds to tens of thousands) after virtualization. Furthermore, the inclusion of the VxLang VM runtime and bytecode resulted in a considerable increase in executable file size, particularly impactful for smaller applications.

The findings highlight a clear trade-off: VxLang provides strong protection against reverse engineering at the cost of significant performance degradation and increased file size. Therefore, its practical application likely requires a selective approach, targeting only the most critical and sensitive code sections where the security benefits outweigh the performance impact.

Future work could involve exploring more advanced reverse engineering techniques specifically targeting VM-based protections to further assess VxLang's resilience. Investigating the impact of different VxLang configuration options on the security-performance balance would also be valuable. Comparative studies with other commercial or open-source virtualization solutions could provide a broader perspective on the state-of-the-art in VM-based obfuscation. Analyzing the security of the VxLang VM interpreter itself could also reveal potential vulnerabilities.
