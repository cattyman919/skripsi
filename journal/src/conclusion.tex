\section{Conclusion} \label{sec:conclusion}
This paper investigated the effectiveness of code virtualization using the VxLang framework as a technique to mitigate software reverse engineering. The process involved marking code with SDK macros, compiling intermediate executables (\texttt{*\_vm.exe}), and processing them directly via the \texttt{vxlang.exe} command-line tool to generate final virtualized binaries (\texttt{*\_vxm.exe}). Through experimental analysis involving static (Ghidra) and dynamic (x64dbg) examination of authentication applications and performance benchmarking (QuickSort, AES), we draw the following conclusions:

VxLang's code virtualization significantly enhances software security by substantially increasing the difficulty of reverse engineering. The transformation into custom bytecode rendered standard static analysis tools ineffective at interpreting program logic and control flow within virtualized sections. Dynamic analysis was similarly obstructed by the VM execution model, making runtime tracing and manipulation arduous. Attempts to bypass authentication logic, which were trivial in non-virtualized versions, were successfully thwarted in the virtualized binaries using the employed techniques.

Furthermore, analysis of a virtualized RAT (Lilith) showed maintained functionality alongside reduced detection rates and a shift from specific signatures to generic flags on VirusTotal, demonstrating VxLang's capability to also evade traditional antivirus detection mechanisms.

However, this robust security comes with significant drawbacks. We observed substantial performance overhead, with execution times for computational tasks increasing dramatically (by factors ranging from hundreds to tens of thousands) after virtualization. Furthermore, the inclusion of the VxLang VM runtime and bytecode resulted in a considerable increase in executable file size, particularly impactful for smaller applications.

The findings highlight a clear trade-off: VxLang provides strong protection against reverse engineering at the cost of significant performance degradation and increased file size. Therefore, its practical application likely requires a selective approach, targeting only the most critical and sensitive code sections where the security benefits outweigh the performance impact.

Future work could involve exploring more advanced reverse engineering techniques specifically targeting VM-based protections to further assess VxLang's resilience. Investigating the impact of different VxLang configuration options on the security-performance balance would also be valuable. Comparative studies with other commercial or open-source virtualization solutions could provide a broader perspective. Investigating the interaction between VxLang and various antivirus detection techniques (signature-based, heuristic, AI/ML, behavioral) would also yield valuable insights into its detection evasion capabilities and limitations.
