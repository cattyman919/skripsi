\documentclass[journal]{IEEEtran}

\usepackage{_internals/preamble}

%-----------------------------------------------------------------------------%
% Informasi Mengenai Dokumen
%-----------------------------------------------------------------------------%
% 
% Judul laporan. 
\var{\judul}{Implementasi dan Analisis Efektivitas \f{Code Virtualization} dalam Meningkatkan Keamanan \f{Software} dengan Mempersulit \f{Reverse Engineering} menggunakan VxLang}
% 
% Tulis kembali judul laporan, kali ini akan diubah menjadi huruf kapital
\Var{\Judul}{Implementasi dan Analisis Efektivitas \f{Code Virtualization} dalam Meningkatkan Keamanan \f{Software} dengan Mempersulit \f{Reverse Engineering} menggunakan VxLang}
% 
% Tulis kembali judul laporan namun dengan bahasa Ingris
\var{\judulInggris}{Implementation and Analysis of the Effectiveness of Code Virtualization in Improving Software Security by complicating Reverse Engineering}
% 
% Tipe laporan, dapat berisi Skripsi, Tugas Akhir, Thesis, atau Disertasi
\var{\type}{Skripsi}
% 
% Tulis kembali tipe laporan, kali ini akan diubah menjadi huruf kapital
\Var{\Type}{Skripsi}
% 
% Tulis nama penulis 
\var{\penulis}{Seno Pamungkas Rahman}
% 
% Tulis kembali nama penulis, kali ini akan diubah menjadi huruf kapital
\Var{\Penulis}{Seno Pamungkass Rahman}
% 
% Tulis NPM penulis
\var{\npm}{2106731586}
% 
% Tuliskan Fakultas dimana penulis berada
\Var{\Fakultas}{Teknik}
\var{\fakultas}{Teknik}
% 
% Tuliskan Program Studi yang diambil penulis
\Var{\Program}{Teknik Komputer}
\var{\program}{Teknik Komputer}
% 
% Tuliskan tahun publikasi laporan
\Var{\bulan}{Januari}
\Var{\tahun}{2025}
% 
% Tuliskan gelar yang akan diperoleh dengan menyerahkan laporan ini
\var{\gelar}{Sarjana Teknik}
% 
% Tuliskan tanggal pengesahan laporan, waktu dimana laporan diserahkan ke 
% penguji/sekretariat
\var{\tanggalPengesahan}{Januari 2025}
% 
% Tuliskan tanggal keputusan sidang dikeluarkan dan penulis dinyatakan 
% lulus/tidak lulus
\var{\tanggalLulus}{Januari 2025}
% 
% Tuliskan pembimbing 
\var{\pembimbing}{Dr. Ruki Harwahyu, S.T. M.T. MSc.}
% 
% Tuliskan penguji
\var{\pengujisatu}{Dr. Ruki Harwahyu, ST. MT. MSc.}
\var{\pengujidua}{I Gde Dharma Nugraha, S.T., M.T., Ph.D}
% 
% Alias untuk memudahkan alur penulisan paa saat menulis laporan
\var{\saya}{Penulis}

%-----------------------------------------------------------------------------%
% Judul Setiap Bab
%-----------------------------------------------------------------------------%
% 
% Berikut ada judul-judul setiap bab. 
% Silahkan diubah sesuai dengan kebutuhan. 
% 
\Var{\kataPengantar}{Kata Pengantar}
\Var{\babSatu}{Pendahuluan}
\Var{\babDua}{Tinjauan Pustaka}
\Var{\babTiga}{Metode Penelitian}
\Var{\babEmpat}{Implementasi}
\Var{\babLima}{Hasil Penelitian}
\Var{\babEnam}{Kesimpulan dan Saran}


\begin{document}
%
% paper title
% Titles are generally capitalized except for words such as a, an, and, as,
% at, but, by, for, in, nor, of, on, or, the, to and up, which are usually
% not capitalized unless they are the first or last word of the title.
% Linebreaks \\ can be used within to get better formatting as desired.
% Do not put math or special symbols in the title.
\title{Bare Demo of IEEEtran.cls\\ for IEEE Journals}
%
%
% author names and IEEE memberships
% note positions of commas and nonbreaking spaces ( ~ ) LaTeX will not break
% a structure at a ~ so this keeps an author's name from being broken across
% two lines.
% use \thanks{} to gain access to the first footnote area
% a separate \thanks must be used for each paragraph as LaTeX2e's \thanks
% was not built to handle multiple paragraphs
%

\author{\penulis~Shell,~\IEEEmembership{Member,~IEEE,}
        John~Doe,~\IEEEmembership{Fellow,~OSA,}
        and~Jane~Doe,~\IEEEmembership{Life~Fellow,~IEEE}% <-this % stops a space
\thanks{M. Shell was with the Department
of Electrical and Computer Engineering, Georgia Institute of Technology, Atlanta,
GA, 30332 USA e-mail: (see http://www.michaelshell.org/contact.html).}% <-this % stops a space
\thanks{J. Doe and J. Doe are with Anonymous University.}% <-this % stops a space
\thanks{Manuscript received April 19, 2005; revised August 26, 2015.}}

% note the % following the last \IEEEmembership and also \thanks - 
% these prevent an unwanted space from occurring between the last author name
% and the end of the author line. i.e., if you had this:
% 
% \author{....lastname \thanks{...} \thanks{...} }
%                     ^------------^------------^----Do not want these spaces!
%
% a space would be appended to the last name and could cause every name on that
% line to be shifted left slightly. This is one of those "LaTeX things". For
% instance, "\textbf{A} \textbf{B}" will typeset as "A B" not "AB". To get
% "AB" then you have to do: "\textbf{A}\textbf{B}"
% \thanks is no different in this regard, so shield the last } of each \thanks
% that ends a line with a % and do not let a space in before the next \thanks.
% Spaces after \IEEEmembership other than the last one are OK (and needed) as
% you are supposed to have spaces between the names. For what it is worth,
% this is a minor point as most people would not even notice if the said evil
% space somehow managed to creep in.



% The paper headers
\markboth{Journal of \LaTeX\ Class Files,~Vol.~14, No.~8, August~2015}%
{Shell \MakeLowercase{\textit{et al.}}: Bare Demo of IEEEtran.cls for IEEE Journals}
% The only time the second header will appear is for the odd numbered pages
% after the title page when using the twoside option.
% 
% *** Note that you probably will NOT want to include the author's ***
% *** name in the headers of peer review papers.                   ***
% You can use \ifCLASSOPTIONpeerreview for conditional compilation here if
% you desire.




% If you want to put a publisher's ID mark on the page you can do it like
% this:
%\IEEEpubid{0000--0000/00\$00.00~\copyright~2015 IEEE}
% Remember, if you use this you must call \IEEEpubidadjcol in the second
% column for its text to clear the IEEEpubid mark.



% use for special paper notices
%\IEEEspecialpapernotice{(Invited Paper)}




% make the title area
\maketitle

% As a general rule, do not put math, special symbols or citations
% in the abstract or keywords.
\begin{abstract}
Reverse engineering poses a significant threat to software security, enabling attackers to analyze, understand, and illicitly modify program code. Code obfuscation techniques, particularly code virtualization, offer a promising defense mechanism. This paper presents an implementation and analysis of the effectiveness of code virtualization using the VxLang framework in enhancing software security against reverse engineering. We applied VxLang's virtualization to critical sections of case study applications, including authentication logic. Static analysis using Ghidra and dynamic analysis using x64dbg were performed on both the original and virtualized binaries. The results demonstrate that VxLang significantly increases the complexity of reverse engineering. Static analysis tools struggled to disassemble and interpret the virtualized code, failing to identify instructions, functions, or meaningful data structures. Dynamic analysis was similarly hampered, with obfuscated control flow and the virtual machine's execution model obscuring runtime behavior and hindering debugging attempts. However, this enhanced security comes at the cost of substantial performance overhead, observed in QuickSort algorithm execution and AES encryption benchmarks, along with a significant increase in executable file size. The findings confirm that VxLang provides robust protection against reverse engineering but necessitates careful consideration of the performance trade-offs for practical deployment.
\end{abstract}


% Note that keywords are not normally used for peerreview papers.
\begin{IEEEkeywords}
IEEE, IEEEtran, journal, \LaTeX, paper, template.
\end{IEEEkeywords}







% For peer review papers, you can put extra information on the cover
% page as needed:
% \ifCLASSOPTIONpeerreview
% \begin{center} \bfseries EDICS Category: 3-BBND \end{center}
% \fi
%
% For peerreview papers, this IEEEtran command inserts a page break and
% creates the second title. It will be ignored for other modes.
\IEEEpeerreviewmaketitle



\section{Introduction}
% The very first letter is a 2 line initial drop letter followed
% by the rest of the first word in caps.
% 
% form to use if the first word consists of a single letter:
% \IEEEPARstart{A}{demo} file is ....
% 
% form to use if you need the single drop letter followed by
% normal text (unknown if ever used by the IEEE):
% \IEEEPARstart{A}{}demo file is ....
% 
% Some journals put the first two words in caps:
% \IEEEPARstart{T}{his demo} file is ....
% 
% Here we have the typical use of a "T" for an initial drop letter
% and "HIS" in caps to complete the first word.
\IEEEPARstart{T}{his} demo file is intended to serve as a ``starter file''
for IEEE journal papers produced under \LaTeX\ using
IEEEtran.cls version 1.8b and later.
% You must have at least 2 lines in the paragraph with the drop letter
% (should never be an issue)
I wish you the best of success.

\hfill mds
 
\hfill August 26, 2015

\subsection{Subsection Heading Here}
Subsection text here.

% needed in second column of first page if using \IEEEpubid
%\IEEEpubidadjcol

\subsubsection{Subsubsection Heading Here}
Subsubsection text here.


% An example of a floating figure using the graphicx package.
% Note that \label must occur AFTER (or within) \caption.
% For figures, \caption should occur after the \includegraphics.
% Note that IEEEtran v1.7 and later has special internal code that
% is designed to preserve the operation of \label within \caption
% even when the captionsoff option is in effect. However, because
% of issues like this, it may be the safest practice to put all your
% \label just after \caption rather than within \caption{}.
%
% Reminder: the "draftcls" or "draftclsnofoot", not "draft", class
% option should be used if it is desired that the figures are to be
% displayed while in draft mode.
%
%\begin{figure}[!t]
%\centering
%\includegraphics[width=2.5in]{myfigure}
% where an .eps filename suffix will be assumed under latex, 
% and a .pdf suffix will be assumed for pdflatex; or what has been declared
% via \DeclareGraphicsExtensions.
%\caption{Simulation results for the network.}
%\label{fig_sim}
%\end{figure}

% Note that the IEEE typically puts floats only at the top, even when this
% results in a large percentage of a column being occupied by floats.


% An example of a double column floating figure using two subfigures.
% (The subfig.sty package must be loaded for this to work.)
% The subfigure \label commands are set within each subfloat command,
% and the \label for the overall figure must come after \caption.
% \hfil is used as a separator to get equal spacing.
% Watch out that the combined width of all the subfigures on a 
% line do not exceed the text width or a line break will occur.
%
%\begin{figure*}[!t]
%\centering
%\subfloat[Case I]{\includegraphics[width=2.5in]{box}%
%\label{fig_first_case}}
%\hfil
%\subfloat[Case II]{\includegraphics[width=2.5in]{box}%
%\label{fig_second_case}}
%\caption{Simulation results for the network.}
%\label{fig_sim}
%\end{figure*}
%
% Note that often IEEE papers with subfigures do not employ subfigure
% captions (using the optional argument to \subfloat[]), but instead will
% reference/describe all of them (a), (b), etc., within the main caption.
% Be aware that for subfig.sty to generate the (a), (b), etc., subfigure
% labels, the optional argument to \subfloat must be present. If a
% subcaption is not desired, just leave its contents blank,
% e.g., \subfloat[].


% An example of a floating table. Note that, for IEEE style tables, the
% \caption command should come BEFORE the table and, given that table
% captions serve much like titles, are usually capitalized except for words
% such as a, an, and, as, at, but, by, for, in, nor, of, on, or, the, to
% and up, which are usually not capitalized unless they are the first or
% last word of the caption. Table text will default to \footnotesize as
% the IEEE normally uses this smaller font for tables.
% The \label must come after \caption as always.
%
%\begin{table}[!t]
%% increase table row spacing, adjust to taste
%\renewcommand{\arraystretch}{1.3}
% if using array.sty, it might be a good idea to tweak the value of
% \extrarowheight as needed to properly center the text within the cells
%\caption{An Example of a Table}
%\label{table_example}
%\centering
%% Some packages, such as MDW tools, offer better commands for making tables
%% than the plain LaTeX2e tabular which is used here.
%\begin{tabular}{|c||c|}
%\hline
%One & Two\\
%\hline
%Three & Four\\
%\hline
%\end{tabular}
%\end{table}


% Note that the IEEE does not put floats in the very first column
% - or typically anywhere on the first page for that matter. Also,
% in-text middle ("here") positioning is typically not used, but it
% is allowed and encouraged for Computer Society conferences (but
% not Computer Society journals). Most IEEE journals/conferences use
% top floats exclusively. 
% Note that, LaTeX2e, unlike IEEE journals/conferences, places
% footnotes above bottom floats. This can be corrected via the
% \fnbelowfloat command of the stfloats package.





\section{Conclusion} \label{sec:conclusion}
This paper investigated the effectiveness of code virtualization using the VxLang framework as a technique to mitigate software reverse engineering. The process involved marking code with SDK macros, compiling intermediate executables (\texttt{*\_vm.exe}), and processing them directly via the \texttt{vxlang.exe} command-line tool to generate final virtualized binaries (\texttt{*\_vxm.exe}). Through experimental analysis involving static (Ghidra) and dynamic (x64dbg) examination of authentication applications and performance benchmarking (QuickSort, AES), we draw the following conclusions:

VxLang's code virtualization significantly enhances software security by substantially increasing the difficulty of reverse engineering. The transformation into custom bytecode rendered standard static analysis tools ineffective at interpreting program logic and control flow within virtualized sections. Dynamically, while x64dbg could observe the native instructions of the VxLang Virtual Machine (VM) itself during runtime, the core application logic remained effectively obscured. Critical strings targeted by virtualization were not discoverable via standard debugger searches, and the abstraction of the application's decision-making processes into the VM's bytecode execution made direct runtime manipulation (e.g., patching conditional jumps for authentication bypass) unfeasible. Attempts to bypass authentication logic, which were trivial in non-virtualized versions, were successfully thwarted in the virtualized binaries using the employed static and dynamic analysis techniques. \textbf{However, the application of these protective measures required careful, iterative placement of virtualization macros, as improper application, particularly in complex code sections involving I/O or intricate control flows, was found to potentially disrupt software functionality, as observed with the Lilith RAT case study.}

Furthermore, analysis of a virtualized RAT (Lilith) showed maintained functionality alongside reduced detection rates and a shift from specific signatures to generic flags on VirusTotal, demonstrating VxLang's capability to also evade traditional antivirus detection mechanisms.

However, this robust security comes with significant drawbacks. We observed substantial performance overhead, with execution times for computational tasks increasing dramatically (by factors ranging from hundreds to tens of thousands) after virtualization. Furthermore, the inclusion of the VxLang VM runtime and bytecode resulted in a considerable increase in executable file size, particularly impactful for smaller applications.

The findings highlight a clear trade-off: VxLang provides strong protection against reverse engineering at the cost of significant performance degradation and increased file size. Therefore, its practical application likely requires a selective approach, targeting only the most critical and sensitive code sections where the security benefits outweigh the performance impact.

Future work could involve exploring more advanced reverse engineering techniques specifically targeting VM-based protections to further assess VxLang's resilience. Investigating the impact of different VxLang configuration options on the security-performance balance would also be valuable. \textbf{Further research into the specific code constructs or patterns that interact poorly with VxLang's virtualization process could yield guidelines for more robust and reliable application of such protection mechanisms.} Comparative studies with other commercial or open-source virtualization solutions could provide a broader perspective. Investigating the interaction between VxLang and various antivirus detection techniques (signature-based, heuristic, AI/ML, behavioral) would also yield valuable insights into its detection evasion capabilities and limitations.






% if have a single appendix:
%\appendix[Proof of the Zonklar Equations]
% or
%\appendix  % for no appendix heading
% do not use \section anymore after \appendix, only \section*
% is possibly needed

% use appendices with more than one appendix
% then use \section to start each appendix
% you must declare a \section before using any
% \subsection or using \label (\appendices by itself
% starts a section numbered zero.)
%


\appendices
\section{Proof of the First Zonklar Equation}
Appendix one text goes here.

% you can choose not to have a title for an appendix
% if you want by leaving the argument blank
\section{}
Appendix two text goes here.


% use section* for acknowledgment
% use section* for acknowledgment
\section*{Acknowledgment}


The authors would like to thank...


% Can use something like this to put references on a page
% by themselves when using endfloat and the captionsoff option.
\ifCLASSOPTIONcaptionsoff
  \newpage
\fi



% trigger a \newpage just before the given reference
% number - used to balance the columns on the last page
% adjust value as needed - may need to be readjusted if
% the document is modified later
%\IEEEtriggeratref{8}
% The "triggered" command can be changed if desired:
%\IEEEtriggercmd{\enlargethispage{-5in}}

% references section

% can use a bibliography generated by BibTeX as a .bbl file
% BibTeX documentation can be easily obtained at:
% http://mirror.ctan.org/biblio/bibtex/contrib/doc/
% The IEEEtran BibTeX style support page is at:
% http://www.michaelshell.org/tex/ieeetran/bibtex/
%\bibliographystyle{IEEEtran}
% argument is your BibTeX string definitions and bibliography database(s)
%\bibliography{IEEEabrv,../bib/paper}
%
% <OR> manually copy in the resultant .bbl file
% set second argument of \begin to the number of references
% (used to reserve space for the reference number labels box)
% Ensure journal.tex has:
\printbibliography


% biography section
% 
% If you have an EPS/PDF photo (graphicx package needed) extra braces are
% needed around the contents of the optional argument to biography to prevent
% the LaTeX parser from getting confused when it sees the complicated
% \includegraphics command within an optional argument. (You could create
% your own custom macro containing the \includegraphics command to make things
% simpler here.)
%\begin{IEEEbiography}[{\includegraphics[width=1in,height=1.25in,clip,keepaspectratio]{mshell}}]{Michael Shell}
% or if you just want to reserve a space for a photo:

\begin{IEEEbiography}{Michael Shell}
Biography text here.
\end{IEEEbiography}

% if you will not have a photo at all:
\begin{IEEEbiographynophoto}{John Doe}
Biography text here.
\end{IEEEbiographynophoto}

% insert where needed to balance the two columns on the last page with
% biographies
%\newpage

\begin{IEEEbiographynophoto}{Jane Doe}
Biography text here.
\end{IEEEbiographynophoto}

% You can push biographies down or up by placing
% a \vfill before or after them. The appropriate
% use of \vfill depends on what kind of text is
% on the last page and whether or not the columns
% are being equalized.

%\vfill

% Can be used to pull up biographies so that the bottom of the last one
% is flush with the other column.
%\enlargethispage{-5in}



% that's all folks
\end{document}



