%-----------------------------------------------------------------------------%
\addChapter{Lampiran 1}
\chapter*{Lampiran 1}
%-----------------------------------------------------------------------------%

\begin{listing}[H]
    \begin{minted}{bash}
    #!/bin/bash

    # Specify the path to your input video file
    input_file="timessquaredaywide.mp4"

    # Specify the output directory
    output_directory="output"

    # Specify the number of times to run the benchmark
    num_runs=100

    # Initialize total time to zero
    total_time=0

    # Loop to run the benchmark command multiple times
    for i in $(seq 1 $num_runs)
    do
        start_time=$(date +%s.%N)
        log_file="$output_directory/log_$i.txt"
        
        echo "Running benchmark $i..."
        
        # Add your ffmpeg compression command here
        ffmpeg -i "$input_file" -c:v libx264 -preset medium -crf 32 -c:a aac -strict -2 "$output_directory/output_$i.mp4" > "$log_file" 2>&1
        
        end_time=$(date +%s.%N)
        elapsed_time=$(awk "BEGIN {print $end_time - $start_time}")
        
        echo "Benchmark $i completed in $elapsed_time seconds. Log saved to $log_file."
        
        # Accumulate the elapsed time
        total_time=$(awk "BEGIN {print $total_time + $elapsed_time}")
    done

    # Calculate the average time
    average_time=$(awk "BEGIN {print $total_time / $num_runs}")

    # Print the total average time
    echo "Total average time for $num_runs benchmarks: $average_time seconds."
    \end{minted}
    \caption{Kode Pengujian Kompresi Video}
    \label{code:kode_pengujian_kompresi_video}
\end{listing}

\begin{listing}[H]
    \begin{minted}{python}
    import os
    import time
    import crc32c


    def benchmark_crc32c(data, iterations):
        # Perform CRC32 calculation
        start_time = time.time()
        for _ in range(iterations):
            crc32c.crc32c(data)
        end_time = time.time()

        # Calculate average time per iteration
        avg_time = (end_time - start_time) / iterations
        return avg_time


    def main():
        # Define parameters
        data_size = 10 * 1024 * 1024  # Size of the data (10 MB)
        num_iterations = 100  # Number of iterations to average the time

        # Generate random data
        data = bytearray(os.urandom(data_size))

        # Run the benchmark
        avg_time_crc32c = benchmark_crc32c(data, num_iterations)
        print(
            f"Average time for CRC32 calculation: {avg_time_crc32c * 1000} ms"
        )  # convert to ms


    main()
    \end{minted}
    \caption{Kode Pengujian Validasi Integritas Data}
    \label{code:kode_pengujian_validasi_integritas_data}
\end{listing}

\begin{listing}
    \begin{minted}{python}
    from cryptography.hazmat.primitives.ciphers import Cipher, algorithms, modes
    from cryptography.hazmat.primitives import padding
    from cryptography.hazmat.backends import default_backend
    import time

    # Generate a random AES key and plaintext
    key = b"sixteenbytekey12"
    plaintext = b"The Advanced Encryption Standard (AES) is a symmetric-key encryption algorithm that has become the de facto standard for securing sensitive data worldwide. Developed by Belgian cryptographers Joan Daemen and Vincent Rijmen, AES was adopted by the U.S. National Institute of Standards and Technology (NIST) in 2001 after a rigorous evaluation process involving numerous submissions from around the globe. AES replaced the aging Data Encryption Standard (DES) and its variants, which were becoming increasingly vulnerable to brute-force attacks due to advances in computing power. AES operates on fixed block sizes of 128 bits and supports key lengths of 128, 192, or 256 bits, with the latter providing the highest level of security. Its sophisticated mathematical structure, based on substitution-permutation networks, makes it highly resistant to various cryptanalytic attacks, including differential and linear cryptanalysis. AES employs multiple rounds of substitution, permutation, and key mixing operations to scramble the plaintext data, making it virtually impossible to decrypt without the correct key. One of the key strengths of AES is its versatility and flexibility. It can be implemented in hardware or software, making it suitable for a wide range of applications, from securing communication channels to protecting data at rest. AES has been extensively studied and scrutinized by cryptographers worldwide, and its strength lies in its ability to withstand known attacks while maintaining high performance. Consequently, AES has been widely adopted in various industries, including government, finance, healthcare, and telecommunications, as well as in protocols and standards such as TLS/SSL, IPsec, and wireless encryption standards like WPA2."


    # Function to perform AES encryption benchmark
    def aes_encrypt_benchmark(key, plaintext):
        backend = default_backend()
        cipher = Cipher(algorithms.AES(key), modes.ECB(), backend=backend)
        encryptor = cipher.encryptor()

        padder = padding.PKCS7(algorithms.AES.block_size).padder()
        padded_plaintext = padder.update(plaintext) + padder.finalize()

        start_time = time.time()
        ciphertext = encryptor.update(padded_plaintext) + encryptor.finalize()
        end_time = time.time()

        return ciphertext, end_time - start_time


    # Function to perform AES decryption benchmark
    def aes_decrypt_benchmark(key, ciphertext):
        backend = default_backend()
        cipher = Cipher(algorithms.AES(key), modes.ECB(), backend=backend)
        decryptor = cipher.decryptor()

        start_time = time.time()
        padded_plaintext = decryptor.update(ciphertext) + decryptor.finalize()
        end_time = time.time()

        unpadder = padding.PKCS7(algorithms.AES.block_size).unpadder()
        plaintext = unpadder.update(padded_plaintext) + unpadder.finalize()

        return end_time - start_time
    \end{minted}
\end{listing}
\begin{listing}
    \begin{minted}{python}
    def run_benchmark():
        # Benchmark AES encryption
        ciphertext, aes_time_encrypt = aes_encrypt_benchmark(key, plaintext)
        # print(f"AES Encryption Time: {(aes_time_encrypt * 1000):.5f} ms")

        # Benchmark AES decryption
        aes_time_decrypt = aes_decrypt_benchmark(key, ciphertext)
        # print(f"AES Decryption Time: {(aes_time_decrypt * 1000):.5f} ms")
        return aes_time_encrypt, aes_time_decrypt


    def main(i):
        iterations = 100

        encrypt_time = 0
        decrypt_time = 0
        for _ in range(iterations):
            encrypt_time_iter, decrypt_time_iter = run_benchmark()
            encrypt_time += encrypt_time_iter
            decrypt_time += decrypt_time_iter

        avg_encrypt_time = encrypt_time / iterations
        avg_decrypt_time = decrypt_time / iterations

        print(f"Run {i+1} Average AES Encryption Time: {(avg_encrypt_time * 1000):.5f} ms")
        print(
            f"Run {i+1} Average AES Decryption Time: {(avg_decrypt_time * 1000):.5f} ms\n"
        )
        
    for i in range(1):
        main(i)
    \end{minted}
    \caption{Kode Pengujian Enkripsi AES}
    \label{code:pengujian_enkripsi_aes}
\end{listing}


\begin{figure}
    \begin{quote}
        The Advanced Encryption Standard (AES) is a symmetric-key encryption algorithm that has become the de facto standard for securing sensitive data worldwide. Developed by Belgian cryptographers Joan Daemen and Vincent Rijmen, AES was adopted by the U.S. National Institute of Standards and Technology (NIST) in 2001 after a rigorous evaluation process involving numerous submissions from around the globe. AES replaced the aging Data Encryption Standard (DES) and its variants, which were becoming increasingly vulnerable to brute-force attacks due to advances in computing power. AES operates on fixed block sizes of 128 bits and supports key lengths of 128, 192, or 256 bits, with the latter providing the highest level of security. Its sophisticated mathematical structure, based on substitution-permutation networks, makes it highly resistant to various cryptanalytic attacks, including differential and linear cryptanalysis. AES employs multiple rounds of substitution, permutation, and key mixing operations to scramble the plaintext data, making it virtually impossible to decrypt without the correct key. One of the key strengths of AES is its versatility and flexibility. It can be implemented in hardware or software, making it suitable for a wide range of applications, from securing communication channels to protecting data at rest. AES has been extensively studied and scrutinized by cryptographers worldwide, and its strength lies in its ability to withstand known attacks while maintaining high performance. Consequently, AES has been widely adopted in various industries, including government, finance, healthcare, and telecommunications, as well as in protocols and standards such as TLS/SSL, IPsec, and wireless encryption standards like WPA2.
    \end{quote}
    \caption{Plaintext Data untuk AES Encryption Benchmark}
    \label{fig:aesPlaintextData}
\end{figure}

\addChapter{Lampiran 2}
\chapter*{Lampiran 2}

%-----------------------------------------------------------------------------%
\section{Instalasi dan Konfigurasi Apache Cloudstack}
%-----------------------------------------------------------------------------%
Dalam instalasi ini \saya\ akan menggunakan \f{Home Network} di IP 192.168.0.0/24. Pertama \saya\ akan melakukan instalasi \f{tools} yang dibutuhkan dengan perintah:

\begin{listing}[H]
    \begin{minted}{bash}
    $ sudo apt-get install -y openntpd openssh-server vim htop tar
    $ sudo apt-get install -y intel-microcode
    \end{minted}
\end{listing}

Perintah ini akan menginstall \f{tools} openntpd, openssh-server, vim, htop, tar, dan intel-microcode. Setelah melakukan instalasi \f{tools} ini kemudian \saya\ akan mengganti password dari user root dengan perintah:

\begin{listing}[H]
    \begin{minted}{bash}
    $ sudo passwd root
    \end{minted}
\end{listing}

Password root user ini nantinya akan digunakan untuk memastikan bahwa Apache CloudStack dapat mengakses komputer host untuk melakukan provisioning \vm dan operasi lainnya.
%-----------------------------------------------------------------------------%
\subsection{Pengaturan Jaringan}
%-----------------------------------------------------------------------------%
\saya\ melakukan pengaturan \f{Linux Bridge} yang akan menghandle jaringan CloudStack \f{public, guest, management} dan \f{storage}. Untuk penyederhanaan \saya\ akan menggunakan satu \f{bridge} yang bernama \textbf{cloudbr0} yang akan digunakan untuk semua jaringan ini. Untuk menginstall \f{tools} yang diperlukan untuk membuat \f{bridge} digunakan perintah:

\begin{listing}[H]
    \begin{minted}{bash}      
    $ sudo apt-get install bridge-utils
    \end{minted}
\end{listing}

Setelah menginstall \f{tools} ini, \saya\ akan mengkonfigurasi \f{bridge} \textbf{cloudbr0} dengan menggunakan netplan. Pertama \saya\ memastikan bahwa seleuruh konfigurasi netplan sekarang di rename dengan menambahkan extensi \textbf{.bak} sebagai backup. Setelah itu \saya\ membuat file konfigruasi netplan dengan perintah:

\begin{listing}[H]
    \begin{minted}{bash}     
    $ sudo nano /etc/netplan/01-netcfg.yaml
    \end{minted}
\end{listing}

Perintah ini akan membuka text editor nano dengan file \texttt{01-netcfg.yaml}. Kemudian \saya\ mengisi file konfigurasi tersebut dengan konfigurasi yang terlihat di kode \ref{code:netplan_config}:

\begin{listing}[H]
    \begin{minted}{yaml}
    network:
        version: 2
        renderer: networkd
        ethernets:
            enp2s0:
                dhcp4: false
                dhcp6: false
                optional: true
        bridges:
            cloudbr0:
                addresses: [192.168.0.10/24]
                routes:
                -   to: default
                    via: 192.168.0.1
                nameservers:
                    addresses: [1.1.1.1,8.8.8.8]
                interfaces: [enp2s0]
                dhcp4: false
                dhcp6: false
                parameters:
                    stp: false
                    forward-delay: 0
    \end{minted}
    \caption{Konfigurasi Netplan untuk Cloudbr0}
    \label{code:netplan_config}
\end{listing}

Dalam konfigurasi ini IP dari \f{bridge} \textbf{cloudbr0} adalah 192.168.0.10/24 dan akan menggunakan DNS 1.1.1.1 dan 8.8.8.8. Setelah melakukan konfigurasi ini \saya\ akan mengaktifkan \f{bridge} \textbf{cloudbr0} dengan perintah:

\begin{listing}[H]
    \begin{minted}{bash}     
    $ sudo netplan generate
    $ sudo netplan apply
    $ sudo reboot
    \end{minted}
\end{listing}

Dengan perintah ini akan dilakukan netplan generate dan apply, setelah itu sistem akan melakukan reboot. Sampai sini pembuatan \f{bridge} \textbf{cloudbr0} sudah selesai.

\subsection{Instalasi Cloudstack}
Pertama \saya\ akan melakukan instalasi \texttt{cloudstack-management} dan juga \texttt{mysql-server}. Database yang akan digunakan untuk Apache CloudStack dalam penelitian ini adalah MySQL.

\begin{listing}[H]
    \begin{minted}{bash}    
    $ sudo mkdir -p /etc/apt/keyrings
    $ wget -O- http://packages.shapeblue.com/release.asc | gpg --dearmor | sudo tee /etc/apt/keyrings/cloudstack.gpg > /dev/null
    
    $ echo deb [signed-by=/etc/apt/keyrings/cloudstack.gpg] http://packages.shapeblue.com/cloudstack/upstream/debian/4.18 / > /etc/apt/sources.list.d/cloudstack.list \\
    $ sudo apt-get update -y
    $ sudo apt-get install cloudstack-management mysql-server
    \end{minted}
\end{listing}

Perintah tersebut digunakan untuk menginstal \f{CloudStack Management Server} versi 4.18 dan MySQL Server. Pertama, perintah akan membuat direktori untuk menyimpan kunci GPG yang digunakan untuk mengotentikasi paket dari repositori CloudStack. Kemudian, kunci GPG tersebut diunduh, didekompresi, dan disimpan dalam direktori yang telah dibuat. Selanjutnya, konfigurasi ditambahkan ke file \texttt{sources.list.d} untuk menandatangani paket-paket dari repositori CloudStack. Setelah itu, daftar paket diperbarui dan \f{CloudStack Management Server} serta MySQL Server diinstal menggunakan perintah apt-get. Selain ini juga akan dilakukan instalasi \f{tools} cloudstack usage dan billing, dengan menggunakan perintah:

\begin{listing}[H]
    \begin{minted}{bash}  
    $ sudo apt-get install cloudstack-usage
    \end{minted}
\end{listing}

\subsection{Konfigurasi Database}
Setelah MySQL server selesai di install \saya\ akan melakukan konfigurasi setting InnoDB di mysql server dengan perintah:

\begin{listing}[H]
    \begin{minted}{bash}      
    $ sudo nano /etc/mysql/mysql.conf.d/mysqld.cnf
    \end{minted}
\end{listing}

Kemudian \saya\ menghapus seluruh isi dari file tersebut dan menggantinya dengan konfigurasi yang terlihat di kode \ref{code:mysql_config}:

\begin{listing}[H]
    \begin{minted}{ini}
    [mysqld]

    server_id = 1
    sql-mode="STRICT_TRANS_TABLES,NO_ENGINE_SUBSTITUTION,ERROR_FOR_DIVISION_BY_ZERO
    NO_ZERO_DATE,NO_ZERO_IN_DATE,NO_ENGINE_SUBSTITUTION"
    innodb_rollback_on_timeout=1
    innodb_lock_wait_timeout=600
    max_connections=1000
    log-bin=mysql-bin
    binlog-format = 'ROW'
    \end{minted}
    \caption{Konfigurasi mysqld.cnf}
    \label{code:mysql_config}
\end{listing}

Setelah selesai melakukan konfigurasi \saya\ merestart MySQL server. Hal ini dilakukan agar mysql menggunakan setting yang terbaru. Untuk melakukan restart MySQL server \saya\ menggunakan perintah:

\begin{listing}[H]
    \begin{minted}{bash}     
    $ systemctl restart mysql
    \end{minted}
\end{listing}

Setelah MySQL server selesai di restart langkah selanjunya adalah melakukan deploy database untuk CloudStack. Deploy server ini harus dilakukan sebagai user root. Deploy dilakukan dengan perintah:

\begin{listing}[H]
    \begin{minted}{bash}       
    $ sudo cloudstack-setup-databases cloud:cloud@localhost --deploy-as=root:password -i 192.168.0.10
    \end{minted}
\end{listing}

IP yang digunakan \saya\ adalah IP dari sistem host.

\subsection{Konfigurasi Storage}
Pada tahap ini \saya\ akan melakukan konfigurasi untuk penyimpanan dari cloudstack. Pertama akan dilakukan instalasi storage driver yang akan digunakan dengan perintah:

\begin{listing}[H]
    \begin{minted}{bash}       
    $ sudo apt-get install nfs-kernel-server quota
    $ echo "/export  *(rw,async,no_root_squash,no_subtree_check)" > /etc/exports
    $ sudo mkdir -p /export/primary /export/secondary
    $ sudo exportfs -a
    \end{minted}
\end{listing}

Perintah \texttt{sudo apt-get install nfs-kernel-server quota} digunakan untuk menginstal perangkat lunak server NFS dan modul Quota pada sistem Linux. Selanjutnya, perintah \texttt{echo "/export *(rw,async,no\_root\_squash,no\_subtree\_check)" > /etc/exports} digunakan untuk menetapkan konfigurasi eksport NFS dengan opsi akses tertentu. Kemudian, perintah \texttt{sudo mkdir -p /export/primary /export/secondary} membuat direktori yang akan diakses oleh klien NFS. Terakhir, perintah \texttt{sudo exportfs -a} mengekspor semua direktori yang telah dikonfigurasi sebelumnya untuk diakses oleh klien NFS. Dengan demikian, rangkaian perintah ini mempersiapkan server NFS untuk berbagi data melalui jaringan. Setelah itu \saya\ melakukan konfigurasi NFS server dengan perintah:

\begin{listing}[H]
    \begin{minted}{bash} 
    $ sed -i -e 's/^RPCMOUNTDOPTS="--manage-gids"$/RPCMOUNTDOPTS="-p 892 --manage-gids"/g' /etc/default/nfs-kernel-server
    $ sed -i -e 's/^STATDOPTS=$/STATDOPTS="--port 662 --outgoing-port 2020"/g' /etc/default/nfs-common
    $ echo "NEED_STATD=yes" >> /etc/default/nfs-common
    $ sed -i -e 's/^RPCRQUOTADOPTS=$/RPCRQUOTADOPTS="-p 875"/g' /etc/default/quota
    $ service nfs-kernel-server restart
    \end{minted}
\end{listing}

Perintah-perintah tersebut digunakan untuk mengubah konfigurasi pada server NFS dan Quota pada sistem Linux. Pertama, menggunakan \texttt{sed}, konfigurasi NFS Mount Daemon dan NFS Stat Daemon diubah untuk menggunakan port yang ditentukan dan mengaktifkan manajemen GID. Selanjutnya, sistem Stat Daemon diaktifkan dengan menambahkan opsi yang sesuai ke file konfigurasi NFS Common. Kemudian, konfigurasi RPC Quota Daemon diubah untuk menggunakan port yang ditentukan. Terakhir, layanan NFS Kernel Server di-restart untuk menerapkan perubahan konfigurasi yang telah dilakukan.

\subsection{Konfigurasi KVM}
Tahap selanjutnya adalah konfigurasi KVM dan cloudstack agent. Pertama \saya\ melakukan install KVM dan CloudStack agent dengan perintah:

\begin{listing}[H]
    \begin{minted}{bash}       
    $ sudo apt-get install qemu-kvm cloudstack-agent
    \end{minted}
\end{listing}

Setelah KVM dan CloudStack agent terinstall \saya\ akan melakukan konfigurasi KVM. Pertama \saya\ akan mengubah konfigurasi \texttt{libvirt} dengan perintah:

\begin{listing}[H]
    \begin{minted}{bash}       
    $ sudo sed -i -e 's/\#vnc_listen.*$/vnc_listen = "0.0.0.0"/g' /etc/libvirt/qemu.conf
    $ sudo echo LIBVIRTD_ARGS=\"--listen\" >> /etc/default/libvirtd
    $ sudo systemctl mask libvirtd.socket libvirtd-ro.socket libvirtd-admin.socket libvirtd-tls.socket libvirtd-tcp.socket
    $ sudo systemctl restart libvirtd
    \end{minted}
\end{listing}

Perintah tersebut digunakan untuk mengkonfigurasi \texttt{libvirt}, sebuah toolkit untuk mengelola virtualisasi di Linux. Pertama, perintah \texttt{sed} digunakan untuk mengubah pengaturan dalam file konfigurasi \texttt{qemu.conf} agar libvirt dapat menerima koneksi VNC dari semua interface jaringan. Kemudian, argumen \texttt{--listen} ditambahkan ke konfigurasi libvirtd melalui file \texttt{/etc/default/libvirtd}, yang membuat libvirt mendengarkan koneksi dari interface jaringan. Selanjutnya, perintah \texttt{systemctl mask} digunakan untuk menonaktifkan unit-unit systemd yang terkait dengan libvirtd, seperti socket-soket yang tidak diperlukan, untuk menghindari penggunaan yang tidak diinginkan. Terakhir, layanan libvirtd di-restart agar perubahan konfigurasi dapat diterapkan.

Selanjutnya \saya\ melakukan konfigurasi default untuk libvirtd dengan perintah:

\begin{listing}[H]
    \begin{minted}{bash}
    $ echo 'listen_tls=0' >> /etc/libvirt/libvirtd.conf
    $ echo 'listen_tcp=1' >> /etc/libvirt/libvirtd.conf
    $ echo 'tcp_port = "16509"' >> /etc/libvirt/libvirtd.conf
    $ echo 'mdns_adv = 0' >> /etc/libvirt/libvirtd.conf
    $ echo 'auth_tcp = "none"' >> /etc/libvirt/libvirtd.conf
    $ systemctl restart libvirtd
    \end{minted}
\end{listing}

Perintah-perintah tersebut digunakan untuk mengkonfigurasi default pada file konfigurasi libvirt (libvirtd.conf). Pertama, dengan menambahkan \texttt{listen\_tls=0}, libvirt diminta untuk tidak mendengarkan koneksi TLS, sehingga mematikan mode tersebut. Kemudian, \texttt{listen\_tcp=1} mengaktifkan libvirt untuk mendengarkan koneksi TCP/IP. Penambahan \texttt{tcp\_port = "16509"} menetapkan port TCP yang akan digunakan oleh libvirt. Selanjutnya, dengan \texttt{mdns\_adv = 0}, mDNS advertise untuk penemuan otomatis layanan libvirt dimatikan. Terakhir, \texttt{auth\_tcp = "none"} mengatur libvirt untuk menerima koneksi TCP/IP tanpa melakukan otentikasi. Setelah perubahan-perubahan ini diterapkan, layanan libvirt di-restart agar konfigurasi baru dapat mulai digunakan. Konfigurasi default ini hanya diperlukan \saya\ saat setup awal, karena saat menambahkan host KVM dalam CloudStack, konfigurasi yang lebih aman menggunakan TLS akan diterapkan secara otomatis.

Kemudian \saya\ menambahkan konfigurasi berikut ini di \texttt{/etc/sysctl.conf} lalu menjalankan \texttt{sysctl -p}. Konfigurasi ini perlu ditambahkan pada host tertentu ditempat menjalankan docker dan service lainnya.

\begin{listing}[H]
    \begin{minted}{bash}       
    $ nano /etc/sysctl.conf
    $ echo 'net.bridge.bridge-nf-call-arptables = 0' >> /etc/sysctl.conf
    $ echo 'net.bridge.bridge-nf-call-iptables = 0' >> /etc/sysctl.conf
    $ sysctl -p
    \end{minted}
\end{listing}

Pada perintah ini pertama, akan dibuka file \texttt{/etc/sysctl.conf} menggunakan teks editor nano. Kemudian, dengan menambahkan baris \texttt{net.bridge.bridge-nf-call-arptables = 0} dan \texttt{net.bridge.bridge-nf-call-iptables = 0} ke dalam file tersebut, Ini akan mematikan fungsi kernel yang biasanya dipanggil saat menggunakan Docker dan jembatan jaringan (bridge). Ini penting karena beberapa layanan, seperti Docker, kadang memerlukan konfigurasi khusus agar berjalan dengan lancar di lingkungan host. Terakhir, dengan menjalankan perintah \texttt{sysctl -p}, perubahan konfigurasi yang telah dibuat akan diterapkan tanpa harus me-reboot sistem, sehingga memungkinkan kernel untuk memuat ulang konfigurasi yang baru diterapkan.

Setelah itu \saya\ akan mengenerate host id dengan uuid, hal ini dapat dilakukan dengan perintah:

\begin{listing}[H]
    \begin{minted}{bash}   
    $ sudo apt-get install uuid
    $ UUID=$(uuid)
    $ echo host_uuid = \"$UUID\" >> /etc/libvirt/libvirtd.conf
    $ systemctl restart libvirtd
    \end{minted}
\end{listing}

Perintah ini digunakan untuk meng-generate UUID \f{(Universally Unique Identifier)} untuk host dan mengintegrasikannya ke dalam konfigurasi \texttt{libvirtd}. Perintah \texttt{UUID=\$(uuid)} digunakan untuk meng-generate UUID secara acak dan menyimpannya dalam variabel \texttt{UUID}. UUID adalah string panjang yang unik secara global dan digunakan untuk mengidentifikasi entitas secara unik di sistem. Kemudian, perintah \texttt{echo host\_uuid = \" \$UUID \"  >> /etc/libvirt/libvirtd.conf} menambahkan baris \texttt{host\_uuid = "<nilai UUID>"} ke dalam file konfigurasi \texttt{libvirtd.conf}. Fungsi baris ini adalah untuk menyimpan UUID host di konfigurasi \texttt{libvirtd}, yang akan digunakan oleh \texttt{libvirt} untuk mengidentifikasi host secara unik dalam lingkungan virtualisasi. Terakhir, perintah \texttt{systemctl restart libvirtd} digunakan untuk merestart layanan libvirtd agar perubahan konfigurasi yang telah dilakukan dapat diterapkan.

\subsection{Konfigurasi Firewall}
Untuk melakukan konfigruasi firewall \saya\ menggunakan perintah:

\begin{listing}[H]
    \begin{minted}{bash}       
    # configure firewall rules to allow useful ports
    $ NETWORK=192.168.0.0/24
    $ iptables -A INPUT -s $NETWORK -m state --state NEW -p udp --dport 111 -j ACCEPT
    $ iptables -A INPUT -s $NETWORK -m state --state NEW -p tcp --dport 111 -j ACCEPT
    $ iptables -A INPUT -s $NETWORK -m state --state NEW -p tcp --dport 2049 -j ACCEPT
    $ iptables -A INPUT -s $NETWORK -m state --state NEW -p tcp --dport 32803 -j ACCEPT
    $ iptables -A INPUT -s $NETWORK -m state --state NEW -p udp --dport 32769 -j ACCEPT
    $ iptables -A INPUT -s $NETWORK -m state --state NEW -p tcp --dport 892 -j ACCEPT
    $ iptables -A INPUT -s $NETWORK -m state --state NEW -p tcp --dport 875 -j ACCEPT
    $ iptables -A INPUT -s $NETWORK -m state --state NEW -p tcp --dport 662 -j ACCEPT
    $ iptables -A INPUT -s $NETWORK -m state --state NEW -p tcp --dport 8250 -j ACCEPT
    $ iptables -A INPUT -s $NETWORK -m state --state NEW -p tcp --dport 8080 -j ACCEPT
    $ iptables -A INPUT -s $NETWORK -m state --state NEW -p tcp --dport 9090 -j ACCEPT
    $ iptables -A INPUT -s $NETWORK -m state --state NEW -p tcp --dport 16514 -j ACCEPT
    
    $ sudo apt-get install iptables-persistent
    \end{minted}
\end{listing}

Perintah tersebut bertujuan untuk mengonfigurasi aturan firewall pada sistem menggunakan iptables, dengan fokus pada pembukaan port-port yang umum digunakan untuk layanan-layanan yang digunakan. Pertama, baris \texttt{NETWORK=192.168.0.0/24} mendefinisikan jaringan yang diizinkan melalui aturan firewall. Selanjutnya, setiap perintah \texttt{iptables -A INPUT ... -j ACCEPT} menambahkan aturan iptables untuk mengizinkan koneksi baru pada port tertentu dari jaringan yang telah ditentukan sebelumnya. Misalnya, perintah \texttt{iptables -A INPUT -s \$NETWORK -m state --state NEW -p udp --dport 111 -j ACCEPT} mengizinkan koneksi UDP baru pada port 111 dari jaringan yang telah ditentukan.

Aturan-aturan tersebut dirancang untuk membuka akses ke port-port yang umum digunakan oleh layanan-layanan seperti NFS, RPC, dan layanan manajemen tertentu. Setelah semua aturan ditambahkan, perintah \texttt{apt-get install iptables-persistent} digunakan untuk menginstal paket \texttt{iptables-persistent}, yang bertujuan agar konfigurasi aturan firewall yang telah dibuat akan dipertahankan dan diterapkan secara otomatis setiap kali sistem di-boot. Dengan demikian, langkah-langkah ini membantu menjaga keamanan sistem dengan memperbolehkan akses hanya ke port-port yang diperlukan dan memastikan bahwa aturan firewall tidak hilang setelah reboot.

Setelah itu kita harus menonaktifkan \texttt{apparmor} pada \texttt{libvirtd}, dikarenakan CloudStack melakukan berbagai macam hal yang memungkinkan bisa terblokir oleh mekanisme pertahanan seperti \texttt{apparmor} \cite{apacheHostInstallation}. Untuk mematikan apparmor \saya\ melakukan perintah:

\begin{listing}[H]
    \begin{minted}{bash}   
    $ ln -s /etc/apparmor.d/usr.sbin.libvirtd /etc/apparmor.d/disable/
    $ ln -s /etc/apparmor.d/usr.lib.libvirt.virt-aa-helper /etc/apparmor.d/disable/
    $ apparmor_parser -R /etc/apparmor.d/usr.sbin.libvirtd
    $ apparmor_parser -R /etc/apparmor.d/usr.lib.libvirt.virt-aa-helper
    \end{minted}
\end{listing}

Perintah-perintah tersebut bertujuan untuk menonaktifkan \texttt{apparmor} pada layanan \texttt{libvirtd} (Libvirt daemon). Langkah pertama adalah membuat symlink (tautan simbolis) dari file konfigurasi \texttt{apparmor} yang berkaitan dengan \texttt{libvirtd} dan helpernya ke dalam direktori \texttt{/etc/apparmor.d/disable/}. Dengan melakukan ini, \texttt{apparmor} diarahkan untuk mengabaikan atau menonaktifkan aturan-aturan yang telah didefinisikan untuk \texttt{libvirtd}. Selanjutnya, perintah \texttt{apparmor\_parser -R} digunakan untuk me-refresh aturan-aturan \texttt{apparmor} yang terkait dengan \texttt{libvirtd} dan helpernya, sehingga aturan-aturan yang telah dinonaktifkan dengan symlink dapat diaplikasikan secara efektif.

Perlu \saya\ ingat bahwa menonaktifkan \texttt{apparmor} dapat meningkatkan risiko keamanan sistem karena menghilangkan lapisan proteksi yang diberikan oleh \texttt{apparmor} terhadap layanan tersebut.

\subsection{Menjalankan CloudStack}
Sampai tahap ini instalasi CloudStack sudah selesai dan untuk menjalankan CloudStack \saya\ menggunakan perintah ini:

\begin{listing}[H]
    \begin{minted}{bash}  
    $ sudo cloudstack-setup-management
    \end{minted}
\end{listing}

Setelah server manajemen sudah UP, CloudStack dapat \saya\ akses di \texttt{http://192.168.0.10:8080} (IP dari \textbf{cloudbr0}) dan masuk dengan kredensial default yaitu admin/admin. Saat masuk pertama kali akan dilanjutkan dengan pengaturan Apache CloudStack. Dikonfigurasi \saya\ akan mengatur Zone, Network, Pod, Cluster, Host, Primary Storage, dan Secondary Storage.

Setelah konfigurasi selesai, \saya\ melihat data di \texttt{Navigation>Infrasturcture>Summary} dan memastikan seluruh komponen sudah dinyalakan. Untuk langkah selanjutnya \saya\ akan menambahkan file iso yang diinginkan untuk membuat instance.