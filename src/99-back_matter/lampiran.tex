%-----------------------------------------------------------------------------%
\chapter*{Kode Sumber Lengkap Aplikasi Autentikasi}
\label{app:kode_sumber} % Label untuk referensi dari Bab 4
%-----------------------------------------------------------------------------%

Berikut adalah kode sumber lengkap untuk beberapa bagian penting dari aplikasi studi kasus autentikasi yang dibahas pada Bab 4.

% --- Kode Aplikasi Konsol Hardcoded ---

\begin{minted}{cpp}
#include <iostream>
#include <string>
#include <vector> // Contoh include lain jika ada

// Jika menggunakan VxLang, header-nya di-include di sini
#ifdef USE_VL_MACRO
#include "vxlang/vxlib.h"
#endif

int main(int, char *[]) {
    std::string inputUsername;
    std::string inputPassword;

    std::cout << "Enter username: ";
    std::cin >> inputUsername;
    std::cout << "Enter password: ";
    std::cin >> inputPassword;

    #ifdef USE_VL_MACRO
    VL_VIRTUALIZATION_BEGIN;
    #endif

    if (inputUsername.compare("seno") == 0 &&
        inputPassword.compare("rahman") == 0) {
        std::cout << "Authorized!" << std::endl;
    } else {
        std::cout << "Not authorized." << std::endl;
    }

    #ifdef USE_VL_MACRO
    VL_VIRTUALIZATION_END;
    #endif

    system("pause"); // Hanya untuk Windows agar console tidak langsung tertutup
    return 0;
}
    \end{minted}
\captionof{listing}{Kode Lengkap: Aplikasi Konsol Varian Hardcoded (\code{console.cpp})}
\label{lst:console_hardcoded_full}

% --- Kode Aplikasi Konsol Cloud ---
\begin{minted}{cpp}
#include <iostream>
#include <string>
#include "cloud.hpp" // Asumsi header ini berisi send_login_request

// Jika menggunakan VxLang, header-nya di-include di sini
#ifdef USE_VL_MACRO
#include "vxlang/vxlib.h"
#endif

int main(int, char *[]) {
    std::string inputUsername;
    std::string inputPassword;

    std::cout << "Enter username: ";
    std::cin >> inputUsername;
    std::cout << "Enter password: ";
    std::cin >> inputPassword;

    bool isAuthenticated = false; // Default value

    #ifdef USE_VL_MACRO
    VL_VIRTUALIZATION_BEGIN;
    #endif

    isAuthenticated = send_login_request(inputUsername, inputPassword);

    if (isAuthenticated) {
        std::cout << "Authorized!" << std::endl;
    } else {
        std::cout << "Not authorized." << std::endl;
    }

    #ifdef USE_VL_MACRO
    VL_VIRTUALIZATION_END;
    #endif

    system("pause"); // Hanya untuk Windows
    return 0;
}
    \end{minted}
\captionof{listing}{Kode Lengkap: Aplikasi Konsol Varian Cloud (\code{console\_cloud.cpp})}
\label{lst:console_cloud_full}


% --- Kode Fungsi send_login_request ---

\begin{minted}{cpp}
#include <string>
#include <iostream> // Untuk debug, bisa dihapus
#include <curl/curl.h>
#include <nlohmann/json.hpp>
#include "cloud.hpp" // Deklarasi fungsi jika dipisah

#ifdef _WIN32
#include <windows.h> // Diperlukan untuk HWND jika digunakan
#else
using HWND = void*; // Placeholder untuk non-Windows
#endif

// Callback function untuk menulis data yang diterima dari cURL ke string
// Sumber: https://curl.se/libcurl/c/getinmemory.html
static size_t WriteCallback(void *contents, size_t size, size_t nmemb, void *userp) {
    size_t realsize = size * nmemb;
    std::string *mem = (std::string*)userp;
    mem->append((char*)contents, realsize);
    return realsize;
}


bool send_login_request(const std::string &username,
                        const std::string &password, HWND /*window*/ /*= NULL*/) {
    CURL *curl;
    CURLcode res;
    std::string readBuffer; // String untuk menyimpan respons server
    std::string url = "http://localhost:9090/login"; // URL server backend

    // Membuat payload JSON
    nlohmann::json payload;
    payload["username"] = username;
    payload["password"] = password;
    std::string jsonStr = payload.dump(); // Konversi JSON ke string

    curl_global_init(CURL_GLOBAL_ALL); // Inisialisasi global libcurl
    curl = curl_easy_init(); // Inisialisasi handle cURL

    if (curl) {
        struct curl_slist *headers = nullptr; // List untuk custom headers
        headers = curl_slist_append(headers, "Content-Type: application/json"); // Set header Content-Type

        // Set opsi cURL
        curl_easy_setopt(curl, CURLOPT_URL, url.c_str()); // Set URL tujuan
        curl_easy_setopt(curl, CURLOPT_HTTPHEADER, headers); // Set custom headers
        curl_easy_setopt(curl, CURLOPT_POSTFIELDS, jsonStr.c_str()); // Set body request (data JSON)
        curl_easy_setopt(curl, CURLOPT_POSTFIELDSIZE, jsonStr.length()); // Set ukuran body
        curl_easy_setopt(curl, CURLOPT_WRITEFUNCTION, WriteCallback); // Set fungsi callback untuk menerima data respons
        curl_easy_setopt(curl, CURLOPT_WRITEDATA, &readBuffer); // Pointer ke string buffer untuk menyimpan respons
        curl_easy_setopt(curl, CURLOPT_TIMEOUT, 5L); // Set timeout request 5 detik
        curl_easy_setopt(curl, CURLOPT_FAILONERROR, 1L); // Gagal jika HTTP status code >= 400

        // Lakukan request HTTP POST
        res = curl_easy_perform(curl);

        // Cleanup
        curl_easy_cleanup(curl);
        curl_slist_free_all(headers); // Bebaskan memori headers

        // Cek hasil request cURL
        if (res != CURLE_OK) {
            // std::cerr << "curl_easy_perform() failed: " << curl_easy_strerror(res) << std::endl;
            curl_global_cleanup();
            return false; // Gagal mengirim request
        }

        // Cek jika buffer kosong (meskipun CURLE_OK, mungkin tidak ada body respons)
        if (readBuffer.empty()) {
           // std::cerr << "Received empty response from server." << std::endl;
            curl_global_cleanup();
            return false;
        }

        // Parse respons JSON
        try {
            nlohmann::json response = nlohmann::json::parse(readBuffer);
            curl_global_cleanup();
            // Kembalikan nilai field "success" dari JSON (default false jika tidak ada)
            return response.value("success", false);
        } catch (const nlohmann::json::parse_error &e) {
            // std::cerr << "JSON parse error: " << e.what() << std::endl;
            // std::cerr << "Received data: " << readBuffer << std::endl;
            curl_global_cleanup();
            return false; // Gagal parse JSON
        }
         catch (const nlohmann::json::exception &e) {
           // std::cerr << "JSON generic error: " << e.what() << std::endl;
           curl_global_cleanup();
           return false; // Error JSON lainnya
        }
    }

    curl_global_cleanup();
    return false; // Gagal menginisialisasi cURL
}
\end{minted}
\captionof{listing}{Kode Lengkap: Implementasi Fungsi \code{send\_login\_request} (\code{cloud.hpp})}
\label{lst:send_login_request_full}

% Tambahkan listing lainnya untuk Qt dan ImGui di sini...

%-----------------------------------------------------------------------------%
\chapter*{Konteks Assembly Analisis Bab 5}
\label{app:kode_asm} % Label untuk referensi dari Bab 5
%-----------------------------------------------------------------------------%

Berikut adalah konteks assembly yang lebih lengkap untuk analisis yang disajikan pada Bab 5.

% --- Konteks Assembly: Analisis Statis Non-Virtualized ---
\begin{minted}{asm}
; --- Konteks Lebih Lengkap untuk Analisis Statis Non-Virtualized (app_imgui) ---
; (Asumsi RDX sudah menunjuk ke input username, RCX sudah menunjuk ke "seno")
; ... (Beberapa instruksi sebelumnya) ...
1400031d6 e8 ...       CALL           VCRUNTIME140.DLL::memcmp ; Cek Username
1400031db 49 83 fe 04  CMP            R14, 0x4                 ; Cek panjang username?
1400031df 74 45        JNZ            LAB_140003226            ; Lompat jika panjang salah
1400031e1 85 c0        TEST           EAX,EAX                  ; Cek hasil username memcmp
1400031e3 74 41        JNZ            LAB_140003226            ; Lompat jika username salah

; --- Lanjut ke cek password jika username OK ---
140003201 48 8d 15 ... LEA            RDX,[s_rahman_140110551] ; Muat alamat "rahman"
140003208 48 89 f9     MOV            RCX,RDI                  ; Muat alamat input password ke RCX
14000320b e8 ...       CALL           VCRUNTIME140.DLL::memcmp ; Cek Password
140003210 48 83 fb 06  CMP            RBX, 0x6                 ; Cek panjang password?
140003214 74 10        JNZ            LAB_140003226            ; Lompat jika panjang salah
140003216 85 c0        TEST           EAX,EAX                  ; Cek hasil password memcmp
140003218 74 0c        JNZ            LAB_140003226            ; Lompat jika password salah

; --- Blok Sukses (jika password benar) ---
; ... (Kode yang dijalankan jika autentikasi berhasil) ...
; ... (Mungkin ada lompatan ke akhir blok auth) ...

LAB_140003226:                                           ; Label Blok Gagal
140003226 48 8b 0d ... MOV            RCX, qword ptr [DAT_140162878] ; Setup MessageBox
14000322d e8 ...       CALL           FUN_140100180            ; Panggil fungsi internal?
140003232 48 8d 15 ... LEA            RDX,[u_Authentication_Failed_1401104da] ; Teks Error
140003239 4c 8d 05 ... LEA            R8,[u_Login_140110506]      ; Judul MessageBox
140003240 48 89 c1     MOV            RCX, RAX                 ; Handle window?
140003243 41 b9 10 ... MOV            R9D, 0x10                ; Tipe MessageBox (MB_ICONERROR)
140003249 ff 15 ...    CALL           qword ptr [->USER32.DLL::MessageBoxW] ; Tampilkan Pesan
; ... (Kelanjutan setelah gagal) ...
    \end{minted}
\captionof{listing}{Konteks Assembly Lengkap: Analisis Statis Non-Virtualized (\textit{app\_imgui})}
\label{lst:asm_static_nonvirt_full}

% --- Konteks Assembly: Analisis Statis Cloud ---
\begin{minted}[linenos=false]{asm}
; --- Konteks Lebih Lengkap untuk Analisis Statis Cloud (console_cloud) ---
; ... (Kode sebelum check hasil auth) ...
LAB_140001754:                                     ; Awal blok check
         140001754 8a 45 bf  MOV      AL, byte ptr [RBP + local_69] ; Ambil hasil
         140001757 a8 01     TEST     AL, 0x1                       ; Cek hasil
         140001759 75 02     JNZ      LAB_14000175d                 ; Lompat jika sukses
         14000175b eb 31     JMP      LAB_14000178e                 ; Lompat ke gagal jika tidak sukses

LAB_14000175d:                                     ; Blok Sukses
         14000175d 48 8b 0d .. MOV      RCX, qword ptr [->MSVCP140D.DLL::std::cout]
         140001764 48 8d 15 .. LEA      RDX, [s_Authorized!_140037270] ; String "Authorized!"
         14000176b e8 d0 0c .. CALL     FUN_140002440                 ; Panggil fungsi print?
         140001770 48 89 45 b0 MOV      qword ptr [RBP + local_78], RAX ; Simpan return value print?
         140001774 eb 00     JMP      LAB_140001776                 ; Lompat ke akhir blok? (Alamat tidak ditampilkan lengkap)

LAB_14000178e:                                     ; Blok Gagal
         14000178e 48 8b 0d .. MOV      RCX, qword ptr [->MSVCP140D.DLL::std::cout]
         140001795 48 8d 15 .. LEA      RDX, [s_Not_authorized_14003727c] ; String "Not authorized"
         14000179c e8 9f 0c .. CALL     FUN_140002440                 ; Panggil fungsi print?
         1400017a1 48 89 45 a8 MOV      qword ptr [RBP + local_80], RAX ; Simpan return value print?
         1400017a5 eb 00     JMP      LAB_1400017a7                 ; Lompat ke akhir blok? (Alamat tidak ditampilkan lengkap)
; ... (Akhir dari blok pengecekan) ...
    \end{minted}
\captionof{listing}{Konteks Assembly Lengkap: Analisis Statis Non-Virtualized Cloud (\textit{console\_cloud})}
\label{lst:asm_static_cloud_full}

% --- Konteks Assembly: Analisis Dinamis Non-Virtualized ---

\begin{minted}[linenos=false]{asm}
; --- Konteks Lebih Lengkap untuk Analisis Dinamis Non-Virtualized (app_qt) ---
; (Asumsi di dalam fungsi handler tombol login)

; Cek Username
LEA rax,qword ptr ds:[<"seno">]      ; Alamat string "seno"
mov qword ptr ss:[rbp+48],rax      ; Simpan alamat ke stack
lea rcx,qword ptr ss:[rbp+A8]      ; Alamat QString input username dari UI
lea rdx,qword ptr ss:[rbp+48]      ; Alamat "seno" di stack
call <app_qt.bool __cdecl operator==(class QString const &, char const *const &)> ; Panggil QString::operator==
mov cl,al                          ; Simpan hasil bool ke CL
xor eax,eax
test cl,1                          ; Cek hasil (Zero Flag set jika false)
mov byte ptr ss:[rbp-41],al        ; Simpan hasil bool ke variabel stack
jne app_qt.7FF7B0724899            ; Lompat jika TIDAK SAMA (ZF=0) ke blok selanjutnya (cek pwd)
jmp app_qt.7FF7B07248B7            ; Lompat jika SAMA (ZF=1) ke blok GAGAL (atau logika lain?)
                                     ; --> Perlu verifikasi ulang logika JNE/JMP ini di debugger

app_qt.7FF7B0724899: ; Label jika username TIDAK SAMA

; Cek Password (Hanya jika username OK - sesuaikan alurnya)
app_qt.7FF7B07248B7: ; (Asumsi label ini adalah jika username SAMA)
LEA rax,qword ptr ds:[<"rahman">]    ; Alamat string "rahman"
mov qword ptr ss:[rbp+40],rax      ; Simpan alamat ke stack
lea rcx,qword ptr ss:[rbp+90]      ; Alamat QString input password dari UI
lea rdx,qword ptr ss:[rbp+40]      ; Alamat "rahman" di stack
call <app_qt.bool __cdecl operator==(class QString const &, char const *const &)> ; Panggil QString::operator==
mov byte ptr ss:[rbp-41],al        ; Simpan hasil bool ke variabel stack
mov al,byte ptr ss:[rbp-41]
test al,1                          ; Cek hasil (Zero Flag set jika false)
jne app_qt.7FF7B07248C3            ; Lompat jika TIDAK SAMA (ZF=0) ke blok SUKSES?
jmp app_qt.7FF7B0724988            ; Lompat jika SAMA (ZF=1) ke blok GAGAL?
                                     ; --> Logika JNE/JMP ini juga perlu diverifikasi

app_qt.7FF7B07248C3: ; (Asumsi Blok Sukses)
; ... (Kode yang dijalankan jika autentikasi berhasil) ...
jmp END_OF_AUTH_CHECK_DYN

# app_qt.7FF7B0724988 (Asumsi Blok Gagal)
lea rcx,qword ptr ss:[rbp+4]       ; 'this' pointer untuk QMessageBox?
mov edx,400                        ; Argumen untuk constructor QFlags?
call <app_qt.public: __cdecl QFlags<...>::QFlags<...>(...)> ; Panggil constructor?
lea rdx,qword ptr ds:[<"Authentication Failed">] ; Alamat string error
; ... (Panggil static method QMessageBox::warning atau critical?) ...

END_OF_AUTH_CHECK_DYN:
; ... (Kelanjutan setelah blok autentikasi) ...
    \end{minted}
\captionof{listing}{Konteks Assembly Lengkap: Analisis Dinamis Non-Virtualized (\textit{app\_qt})}
\label{lst:asm_dynamic_nonvirt_full}

% --- Konteks Assembly: Perbandingan IO Virtualized vs Non-Virtualized ---
\begin{minted}[linenos=false]{asm}
; --- Konteks Lebih Lengkap: Perbandingan Input/Output ---

; --- Non-Virtualized (console) ---
call <console.public: __cdecl std::basic_string<char, struct std::char_traits<char>, class std::allocator<char>>::basic_string<char, struct std::char_traits<char>, class std::allocator<char>>(void)> ; Constructor string?
mov rcx, qword ptr ds:[<class std::basic_ostream<char, struct std::char_traits<char>> std::cout>] ; Pointer ke cout
lea rdx, qword ptr ds:[<"Enter username: ">] ; Pointer ke string prompt
call <console.class std::basic_ostream<char, struct std::char_traits<char>> & __cdecl std::operator<<<struct std::char_traits<char>>(class std::basic_ostream<char, struct std::char_traits<char>> &, char const *)> ; Panggil cout << prompt
jmp console.7FF68CD5104D ; Lompatan dalam fungsi?
mov rcx, qword ptr ds:[<class std::basic_istream<char, struct std::char_traits<char>> std::cin>] ; Pointer ke cin
lea rdx, qword ptr ss:[rbp-28] ; Pointer ke buffer string di stack
call <console.class std::basic_istream<char, struct std::char_traits<char>> & __cdecl std::operator>><char, struct std::char_traits<char>, class std::allocator<char>>(class std::basic_istream<char, struct std::char_traits<char>> &, class std::basic_string<char, struct > ; Panggil cin >> buffer
jmp console.7FF68CD5105F ; Lompatan dalam fungsi?
mov rcx, qword ptr ds:[<class std::basic_ostream<char, struct std::char_traits<char>> std::cout>] ; Pointer ke cout
lea rdx, qword ptr ds:[<"Enter password: ">] ; Pointer ke string prompt
call <console.class std::basic_ostream<char, struct std::char_traits<char>> & __cdecl std::operator<<<struct std::char_traits<char>>(class std::basic_ostream<char, struct std::char_traits<char>> &, char const *)> ; Panggil cout << prompt
; ... (Lanjut baca password) ...


; --- Virtualized (console_vm) ---
00007FF74CD48240 | 40 53                   | push rbx
00007FF74CD48242 | 48 83 EC 20             | sub rsp,20
00007FF74CD48246 | E8 15F00400             | call console_vm.vxm.7FF74CD97260 ; Panggilan internal VM?
00007FF74CD4824B | ??                      | ??? ; Byte tidak valid/dikenali
00007FF74CD4824C | 3026                    | xor byte ptr ds:[rsi],ah
00007FF74CD4824E | 8F                      | ??? ; Byte tidak valid/dikenali
00007FF74CD4824F | EB 03                   | jmp short console_vm.vxm.7FF74CD48254
00007FF74CD48251 | DDD8                    | fstp st(0)
00007FF74CD48253 | 0000                    | add byte ptr ds:[rax],al
00007FF74CD48255 | 0000                    | add byte ptr ds:[rax],al
; ... (Urutan instruksi yang diobfuskasi berlanjut) ...
    \end{minted}
\captionof{listing}{Konteks Assembly Lengkap: Perbandingan Operasi Input/Output}
\label{lst:asm_dynamic_io_comparison_full}

% Tambahkan listing assembly lainnya jika diperlukan...
