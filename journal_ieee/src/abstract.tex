\begin{abstract}
Reverse engineering poses a significant threat to software security, enabling attackers to analyze, understand, and illicitly modify program code. Code obfuscation techniques, particularly code virtualization, offer a promising defense mechanism. This paper presents an implementation and analysis of the effectiveness of code virtualization using the VxLang framework in enhancing software security against reverse engineering. We applied VxLang's virtualization to critical sections of case study applications, including authentication logic. Static analysis using Ghidra and dynamic analysis using x64dbg were performed on both the original and virtualized binaries. The results demonstrate that VxLang significantly increases the complexity of reverse engineering. Static analysis tools struggled to disassemble and interpret the virtualized code, failing to identify instructions, functions, or meaningful data structures. Dynamic analysis was similarly hampered, with obfuscated control flow and the virtual machine's execution model obscuring runtime behavior and hindering debugging attempts. Analysis extended to a Remote Administration Tool (RAT) demonstrated maintained functionality post-virtualization and significantly altered malware detection profiles on VirusTotal, indicating successful evasion of signature-based detection. However, this enhanced security comes at the cost of substantial performance overhead, observed in QuickSort algorithm execution and AES encryption benchmarks, along with a significant increase in executable file size. The findings confirm that VxLang provides robust protection against reverse engineering but necessitates careful consideration of the performance trade-offs for practical deployment.
\end{abstract}
