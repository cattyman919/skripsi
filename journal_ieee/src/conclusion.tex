\section{Conclusion} \label{sec:conclusion}
This paper investigated the effectiveness of code virtualization using the VxLang framework as a technique to mitigate software reverse engineering. The process involved marking code with SDK macros, compiling intermediate executables (\texttt{*\_vm.exe}), and processing them directly via the \texttt{vxlang.exe} command-line tool to generate final virtualized binaries (\texttt{*\_vxm.exe}). Through experimental analysis involving static (Ghidra) and dynamic (x64dbg) examination of authentication applications and performance benchmarking (QuickSort, AES), we draw the following conclusions:

VxLang's code virtualization significantly enhances software security by substantially increasing the difficulty of reverse engineering. The transformation into custom bytecode rendered standard static analysis tools ineffective at interpreting program logic and control flow within virtualized sections. Dynamically, while x64dbg could observe the native instructions of the VxLang Virtual Machine (VM) itself during runtime, the core application logic remained effectively obscured. Critical strings targeted by virtualization were not discoverable via standard debugger searches, and the abstraction of the application's decision-making processes into the VM's bytecode execution made direct runtime manipulation (e.g., patching conditional jumps for authentication bypass) unfeasible. Attempts to bypass authentication logic, which were trivial in non-virtualized versions, were successfully thwarted in the virtualized binaries using the employed static and dynamic analysis techniques. \textbf{However, the application of these protective measures required careful, iterative placement of virtualization macros, as improper application, particularly in complex code sections involving I/O or intricate control flows, was found to potentially disrupt software functionality, as observed with the Lilith RAT case study.}

Furthermore, analysis of ten malware/PUA samples on VirusTotal showed that VxLang's impact on detection rates is varied: approximately half of the samples exhibited a decrease in detections, often with a shift to more generic or heuristic-based flags, while the other half showed an increase in detections, suggesting the virtualization layer itself can trigger alerts. This indicates that while static signatures may be obscured, overall detection evasion is not guaranteed and is highly sample-dependent.

However, this robust security comes with significant drawbacks. We observed substantial performance overhead, with execution times for computational tasks increasing dramatically (by factors ranging from hundreds to tens of thousands) after virtualization. Furthermore, the inclusion of the VxLang VM runtime and bytecode resulted in a considerable increase in executable file size, particularly impactful for smaller applications.

The findings highlight a clear trade-off: VxLang provides strong protection against reverse engineering at the cost of significant performance degradation and increased file size. Therefore, its practical application likely requires a selective approach, targeting only the most critical and sensitive code sections where the security benefits outweigh the performance impact.

Future work could involve exploring more advanced reverse engineering techniques specifically targeting VM-based protections to further assess VxLang's resilience. Investigating the impact of different VxLang configuration options on the security-performance balance would also be valuable. \textbf{Further research into the specific code constructs or patterns that interact poorly with VxLang's virtualization process could yield guidelines for more robust and reliable application of such protection mechanisms.} Comparative studies with other commercial or open-source virtualization solutions could provide a broader perspective. Investigating the interaction between VxLang and various antivirus detection techniques (signature-based, heuristic, AI/ML, behavioral) would also yield valuable insights into its detection evasion capabilities and limitations.
